\documentclass[main.tex]{subfiles}
\begin{document}

\recipe{Muffiny z~mąki ryżowej}

\begin{Ingred}
    \item \num{1} jajko
    \item \num{5} łyżek jogurtu naturalnego
    \item \num{.25} szklanki oleju
    \item \num{2} łyżki cukru trzcinowego
    \item \num{1} płaska łyżeczka proszku do pieczenia
    \item \num{1} szklanka mąki ryżowej
    \item gęsty dżem
    \item lukier z~cukru trzcinowego i~soku z~limonki/cytryny
\end{Ingred}

Osobno wymieszać suche i~mokre składniki, następnie połączyć i~dokładnie
wymieszać. Wypełnić silikonowe formy do muffin. Piec w~piekarniku nagrzanym do
\qty{180}{\celsius} przez około \qtyrange{20}{25}{\minute}.

\recipe{Czekoladowe muffiny bezglutenowe}

\begin{Ingred}
    \item \num{1} duży banan
    \item \qty{140}{\gram} jogurtu naturalnego
    \item \num{1} jajko
    \item \num{4} łyżki miodu
    \item \num{1.5} szklanki mielonych migdałów
    \item \num{2} łyżki mąki ryżowej
    \item \num{1} łyżeczka sody oczyszczonej
    \item \num{2} łyżki kakao
    \item \num{4} łyżki wiórków kokosowych
    \item \num{4} łyżki posiekanych orzechów włoskich
    \item \qty{50}{\gram} czekolady mlecznej lub gorzkiej
\end{Ingred}

Piekarnik nagrzać do \qty{175}{\celsius}. Formę na muffinki wypełnić \num{9}
papilotkami. W misce rozgnieść banana z~jogurtem. Dodać jajko, miód i~wymieszać
całość. W drugiej misce wymieszać mielone migdały, mąkę ryżową, sodę
oczyszczoną, kakao, wiórki kokosowe i~posiekane orzechy włoskie. Do suchych
składników dodać mokre i~wymieszać całość do połączenia składników. Ciasto
nałożyć do papilotek, posypać posiekaną czekoladą i~piec \qty{25}{\minute}.

\recipe{Dyniowe muffiny bezglutenowe}

\begin{Ingred}
    \item \num{1} mała dynia
    \item \num{4} łyżki syropu klonowego
    \item \num{1} jajko
    \item \num{4} łyżki roztopionego oleju kokosowego
    \item \qty{50}{\gram} mąki owsianej
    \item \qty{80}{\gram} mąki ryżowej
    \item \num{1} łyżka kakao
    \item \num{4} łyżki wiórków kokosowych
    \item \num{.5} łyżeczki proszku do pieczenia
    \item \num{.5} łyżeczki sody
    \item \num{.25} łyżeczki imbiru
    \item \num{.5} łyżeczki ekstraktu z~wanilii
    \item \num{1} łyżeczka przyprawy piernikowej
    \item \num{.5} łyżeczki cynamonu
    \item garść posiekanych orzechów włoskich
    \item szczypta soli morskiej
\end{Ingred}

Małą dynię pozbawioną nasion pokroić na mniejsze kawałki, włożyć do piekarnika
nagrzanego do \qty{175}{\celsius} i~piec do momentu gdy można bezproblemowo
wbić widelec. Miąższ oddzielić od skórki, poczekać do wystygnięcia i~zmiksować
na mus. Odmierzyć szklankę do reszty przepisu.

Formę na muffiny wypełnić \num{6} papilotami. W misce wymieszać mokre
składniki, mus z~dyni, syrop klonowy, jajko i~olej kokosowy. W drugiej misce
wymieszać dokładnie suche składniki. Następnie delikatnie wymieszać suche
i~mokre składniki. Nałożyć do papilotek i~piec około \qty{20}{\minute}.

\recipe{Babeczki piernikowe}

\begin{Ingred}
    \item \num{1.5} banana
    \item \qty{125}{\milli\litre} oleju z~pestek winogron
    \item \qty{75}{\gram} cukru trzcinowego
    \item \qty{250}{\milli\litre} passaty pomidorowej
    \item \qty{50}{\gram} mąki gryczanej
    \item \qty{150}{\gram} mąki ryżowej
    \item \num{.5} łyżeczki sody oczyszczonej
    \item \num{1} łyżeczka proszku do pieczenia
    \item \num{2} łyżeczki przyprawy do piernika
    \item szczypta soli
\end{Ingred}

Piekarnik nagrzać do \qty{180}{\celsius}. W misce zblendować banana z~olejem
oraz cukrem trzcinowym. Dodać passaty pomidorowej i~dokładnie wymieszać. Mąkę
gryczaną i~ryżową wymieszać z~sodą oczyszczoną, proszkiem do pieczenia, solą
i~przyprawą do piernika. Wymieszać mokre i~suche składniki. Przełożyć do
foremek i~piec \qty{20}{\minute}.

\recipe{Muffiny kukurydziane}

\begin{Ingred}
    \item \num{3} łyżki ziaren mielonego siemienia lnianego
    \item \num{2} szklanki mąki kukurydzianej
    \item \num{3} łyżki mąki ziemniaczanej
    \item \num{2} łyżeczki proszku do pieczenia
    \item \num{3} łyżki cukru trzcinowego
    \item \num{1} szklanka mleka
    \item \qty{125}{\milli\litre} jogurtu
    \item \qty{50}{\milli\litre} rozpuszczonego masła lub oleju z~pestek winogron
    \item \num{1} duża pomarańcza
    \item \qty{100}{\gram} malin
\end{Ingred}

Piekarnik nagrzać do \qty{180}{\celsius}. W młynku do kawy zmielić \num{3}
łyżki ziaren siemienia lnianego. Przesypać do większej miski i~dodać mąkę
kukurydzianą, ziemniaczaną, proszek do pieczenia i~cukier trzcinowy. W drugiej
misce wymieszać mokre składniki oraz sok wyciśnięty z~pomarańczy oraz startą
skórkę. Połączyć zawartość misek. Nałożyć po łyżeczce masy do papilotek, dodać
po \num{3} maliny i~dopełnić ciastem do \num{.66} wysokości. Na górze dodajemy
dodatkowe maliny. Piec około \qtyrange{20}{25}{\minute}.

\recipe{Blok czekoladowy}
\begin{Ingred}
    \item \num{1} margaryna \enquote{Palma}
    \item \num{.5} szklanki wody
    \item \num{1} cukier waniliowy lub zapach waniliowy
    \item \num{1.5} szklanki cukru
    \item \num{3} łyżki kakao
    \item \num{1} opakowanie mleka w~proszku
    \item \numrange{2}{3} paczki herbatników
    \item bakalie
\end{Ingred}

Margarynę, wodę, cukier waniliowy, cukier i~kakao zagotować, cały czas
mieszając. Zostawić do ostudzenia. Dodać następnie pozostałe składniki,
wymieszać, włożyć do formy i~odstawić do lodówki do wystygnięcia.

\recipe{Sernik od Grażyny W.}

\begin{Ingred}
    \item \qty{1}{\kilo\gram} sera białego
    \item \num{2} budynie śmietankowe
    \item \qty{.5}{\litre} zimnego mleka
    \item \num{1} szklanka cukru
    \item \num{.5} puszki brzoskwiń
    \item \num{1} paczka biszkoptów
    \item wiórki kokosowe
\end{Ingred}

Żółtka utrzeć z~cukrem, dodać przemielony ser i~budyń z~mlekiem, wrzucić kawałki
brzoskwiń. Wylać na blachę wyłożoną biszkoptami. Ubić pianę z~białek, posypać
wiórkami, piec około \qty{1}{\hour}.

\recipe{Ciasto marchewkowe}

\begin{Ingred}[polewa]
    \item \qty{400}{\gram} cukru pudru
    \item \qty{100}{\gram} kremowe serka
    \item \qty{50}{\gram} masła
\end{Ingred}

\begin{Ingred}[ciasto]
    \item \num{2} jaja
    \item \qty{200}{\gram} brązowego cukru
    \item \qty{150}{\milli\litre} oleju roślinnego
    \item \qty{200}{\gram} drobno startej marchewki
    \item \qty{50}{\gram} posiekanych orzechów włoskich
    \item \qty{75}{\gram} drobno pokrojonego ananasa
    \item \qty{50}{\gram} wiórków kokosowych
    \item \qty{200}{\gram} mąki
    \item \num{1} łyżeczka cynamonu
    \item \num{1} łyżeczka sody
    \item \num{1} łyżeczka soli
    \item \num{.5} łyżeczki proszku do pieczenia
\end{Ingred}

\paru{Polewa} Mikserem utrzeć serek z~masłem. Dodać cukier puder, w~trzech
częściach, cały czas ucierając pomiędzy kolejnymi partiami. Wstawić do lodówki,
by polewa lekko stężała.

\paru{Ciasto} Ubić jajka do podwojenia objętości, dodać cukier i~ubijać aż masa
będzie gładka i~puszysta. Ubijać dalej na wysokich obrotach, dolewając ciągłym,
cieniutkim strumieniem olej. Do powstałej masy dodać marchewkę, ananasa,
orzechy, wiórki kokosowe i~dalej mieszać. Dodać przesianą mąkę, cynamon, sodę,
proszek do pieczenia i~sól, delikatnie mieszać. Przełożyć do~małej formy
\qtyproduct{21 x 21}{\centi\metre}, wyłożonej papierem do pieczenia. Piec przez
\qty{1}{\hour} w~piekarniku nagrzanym do \qty{150}{\celsius}. Wystudzone
ciasto, przekroić poziomo na \num{2} części, spód posmarować \num{.33} ilości
polewy. Przykryć górą ciasta i~posmarować resztą polewy, wstawić do lodówki.

\recipe{Ciasto Hali}

\begin{Ingred}[ciasto]
    \item \num{2} szklanki mąki
    \item \num{2} łyżeczki proszku do pieczenia
    \item \num{.5} kostki miękkiej margaryny
    \item \num{.5} szklanki cukru
    \item \num{2} jaja
    \item \num{1} żółtko
\end{Ingred}

\begin{Ingred}[masa budyniowa]
    \item \qty{1}{\litre} mleka
    \item \num{3} budynie czekoladowe
\end{Ingred}

\begin{Ingred}[masa serowa]
    \item \num{.5} kostki miękkiej margaryny
    \item \num{.5} szklanki cukru
    \item \num{4} serki waniliowe
    \item \num{3} jaja
    \item \num{1} białko
    \item \num{1} budyń śmietankowy
    \item polewa czekoladowa
    \item wiórki kokosowe
\end{Ingred}

Zagnieść ciasto i~wyłożyć je na prostokątną blaszkę. Ugotować budynie
czekoladowe, gorące przełożyć na surowe ciasto i~równo rozprowadzić. Odstawić
do wystudzenia.

Margarynę utrzeć z~cukrem i~stale ucierając dodawać na przemian serek waniliowy
i~żółtka. Następnie wsypać budyń śmietankowy i~wymieszać. Na końcu lekko
mieszając, połączyć ze sztywną pianę z~czterech białek i~równo rozłożyć na
budyniu czekoladowym.

Wstawić na \qty{40}{\minute} do rozgrzanego piekarnika \qty{180}{\celsius}. Po
upieczeniu polać polewą i~posypać wiórkami kokosowymi.

\recipe{Ożyrok dla dorosłych}

\begin{Ingred}
    \item \num{1 }szklanka kaszy manny (grysiku)
    \item \num{1} szklanka czystej wódki
    \item \num{1} kostka margaryny
    \item \qty{20}{\deca\gram} cukru pudru
    \item \qty{20}{\deca\gram} wiórki kokosowych
    \item polewa czekoladowa
\end{Ingred}

Na blaszce średniej wielkości upiec biszkopt. Kaszę mannę zalać szklankę wódki,
przykryć i~odstawić na noc, aby kaszka dobrze wchłonęła alkohol. Margarynę
utrzeć z~cukrem pudrem i~dodawać po łyżce namoczonej kaszki. Na koniec dodać
kokos i~wszystko dokładnie wymieszać. Tak przygotowaną masę wyłożyć na
wcześniej upieczony biszkopt i~zalać polewą czekoladową. Ciasto wstawić do
lodówki.

\recipe{Ciasto z~alkoholem}

\begin{Ingred}[biszkopt]
    \item \num{5} jajek
    \item \num{5} łyżek cukru
    \item \num{5} czubatych łyżek mąki
    \item \num{.5} łyżeczki proszku do pieczenia
    \item tłuszcz do wysmarowania blachy
\end{Ingred}

\begin{Ingred}[masa]
    \item \num{1} szklanka kaszy manny
    \item \num{1} szklanka wódki
    \item \num{1} kostka masła
    \item \qty{25}{\deca\gram} cukru pudru
    \item \qty{20}{\deca\gram} wiórków kokosowych
    \item kakao
\end{Ingred}

Ze składników na biszkopt przygotować ciasto, wyłożyć na blachę i~upiec jak
zwykły biszkopt.

Kaszkę zalać wódkę i~odstawić na \qty{3}{\hour}. Do namoczonej kaszki dodać
cukier oraz masło, wszystko dokładnie utrzeć. Na koniec dodać kakao i~dokładnie
wymieszać. Masę wyłożyć na biszkopt, wyrównać powierzchnię i~udekorować według
uznania.

\recipe{Szarlotka Tutti-Frutti}

\begin{Ingred}
    \item \num{5} jaj
    \item \num{3} szklanki mąki
    \item \num{1.5} szklanki cukru
    \item \num{1} kostka margaryny
    \item \num{1.5} łyżeczki proszku do pieczenia
    \item \num{2} łyżeczki mąki ziemniaczanej
    \item pokrojone jabłka lub powidła ewentualnie inne świeże owoce lub z~kompotu, ale
    niezbyt słodkie.
\end{Ingred}

Z żółtek, połowy szklanki cukru, mąki wymieszanej z~proszkiem i~tłuszczy szybko
zagnieść ciasto. Podzielić na trzy części i~dwie z~nich zamrozić. Ubić sztywną
pianę z~białek, wciąż obijając dodać szklankę cukru i~mąkę ziemniaczaną. Nie
zmrożonym ciastem wylepić blachę, wyłożyć owoce, następnie warstwę starego na
tarce zamrożonego ciasta, na to wylać pianę, a~na koniec trzecią część ciasta,
również startego na tarce.

\recipe{Biszkopt z~makiem}

\begin{Ingred}[biszkopt]
    \item \num{4} jajka
    \item \num{1} szklanka mąki
    \item \num{1} płaska łyżeczka proszku do pieczenia
    \item niepełna szklanka cukru
\end{Ingred}

\begin{Ingred}[masa]
    \item \qty{30}{\deka\gram} mąki
    \item \num{1} szklanka cukru
    \item \num{1} kostka margaryny
    \item \num{3} jajka
    \item olejek migdałowy lub rodzynki
    \item orzechy
    \item migdały
    \item polewa
    \item wiórki kokosowe
\end{Ingred}

Jaja ubijać z~cukrem, aż powstanie masa o~konsystencji o~gęstej śmietany,
wsypać mąkę z~proszkiem do pieczenia, delikatnie wymieszać, wylać do tortownicy
i~upiec.

Mak sparzyć, przekręcić \num{3} razy przez maszynkę do mięsa, wymieszać
z~jajkami. Margarynę zagotować z~cukrem, dodać mak, pogotować chwilę dodać
olejek lub bakalie.

Biszkopt przekroić na dwa płaty i~je nasączyć, np. ostudzoną mocną herbatą
z~kieliszkiem alkoholu. Na jedną część biszkoptu, nałożyć połowę masy, przykryć
drugim krążkiem biszkoptu, wyłożyć pozostałą masę i~rozsmarować. Polać polewą,
posypać wiórkami kokosowymi.

\recipe{Fale Dunaju}

\begin{Ingred}[ciasto]
    \item \num{1} kostka masła roślinnego
    \item \num{6} jajek
    \item \num{1} szklanka cukru pudru
    \item \num{2} szklanki mąki
    \item \num{2} łyżki kakao
    \item \num{2} łyżeczki proszku do pieczenia
    \item cukier waniliowy
    \item olejek rumowy
    \item owoce z~kompostu lub świeże
\end{Ingred}

\begin{Ingred}[krem]
    \item \num{1} kostka margaryny
    \item \qty{.5}{\litre} mleka
    \item \num{.5} szklanki cukru
    \item \num{2} łyżki mąki ziemniaczanej
    \item \num{2} łyżki pszennej
    \item cukier waniliowy
\end{Ingred}

\begin{Ingred}[polewa]
    \item \num{.5} kostki margaryny
    \item \num{1} szklanka cukru pudru
    \item \num{2} łyżki kakao
    \item \num{2} łyżki wody
\end{Ingred}

Utrzeć masło z~cukrem, dodając po jednym żółtku, mąkę wymieszaną z~proszkiem
i~ubite na sztywno pianą. Ciasto podzielić na dwie części. Jedną wyłożyć na
tackę, do drugiej dodać kakao i~dwie łyżki wody, po czym wyłożyć na jasną część
ciasta. Na wierzchu dość gęsto wyłożyć owoce i~piec około \qty{45}{\minute}.

Z mleka, cukru, mąki o~utartej margaryny przygotować krem. Gdy ciasto ostygnie
nałożyć na nie krem. Składniki na polewę rozpuścić w~garnku, gdy przestygnie
oblać nią ciasto.

\recipe{Pleśniak}

\begin{Ingred}
    \item \num{6} żółtek
    \item \num{4.5} szklanki mąki
    \item \num{1.5} kostki margaryny
    \item \num{1.5} łyżeczki proszku do pieczenia
    \item \num{4.5} łyżeczki cukru
    \item \numrange{2}{3} łyżki kakao
    \item \num{1} słoik dżemu
    \item \num{6} białek
    \item \num{1} szklanka cukru
    \item cukier waniliowy
\end{Ingred}

Z żółtek, cukru waniliowego, mąki, margaryny, proszku do pieczenia i~\num{4.5}
łyżeczki cukru zagnieść ciasto i~podzielić na \num{3} części. Pierwszą część
ciasta wyłożyć na blachę i~posmarować dżemem. Drugą część wymieszać z~kakao,
pokruszyć na dżem. Białko ubić z~szklanką cukru, wyłożyć na ciasto. Ostatnią
część zetrzeć na tarce i~położyć na pianę.

\recipe{Lodowiec}

\begin{Ingred}
    \item \num{2} paczki biszkoptów okrągłych
    \item \num{2} galaretki
    \item \num{1} szklanka cukru
    \item \num{1} szklanka mleka
    \item \num{4} żółtka
    \item \num{.5} opakowanie budynku śmietankowego
    \item \num{1} kostka masła
    \item kakao
\end{Ingred}

Mleko zagotować i~ostudzić. Żółtka utrzeć z~cukrem i~budyniem. Wymieszać i~wlać
do przestudzonego mleka. Zagotować całość, ciągle mieszając, wystudzić i~utrzeć
z~kostką masła. Podzielić krem na dwie części, do jednej z~nich dodać kakao.
Galaretki rozpuścić w~\num{3} szklankach wody i~wystudzić. W brytfance ułożyć
ścisło warstwę biszkoptów zamoczonych w~galaretce i~przykryć je białym kremem.
Następnie ułożyć drugą warstwę z~biszkoptów zamoczonych w~galaretce i~nałożyć
na nie ciemny krem, dokładnie wypełniając każde szparki. Ciasto odstawić
w~chłodne miejsce. Gdy zastygnie pokroić na kwadraty.

\recipe{Krem do dyskoteki}

\begin{Ingred}
    \item \num{1} szklanka cukru
    \item \num{5} jajek
    \item \num{.25} szklanki wody
    \item \num{1} kostka masła roślinnego
    \item \num{3} różne galaretki
    \item \num{2.5} łyżki żelatyny
\end{Ingred}

Cukier puder, żółtka, masło utrzeć. Galaretki rozpuścić, pokroić w~kostkę.
Żelatynę rozpuścić w~\num{.25} szklanki wody. Chłodne dodać do utartej masy
razem z~ubitymi białkami na pianę. Wszystko wymieszać i~wyłożyć na biszkopt.

\recipe{Polewa od Babci}

\begin{Ingred}
    \item \num{.5} kostki margaryny
    \item \num{5} łyżek cukru pudru
    \item \num{3} łyżki kakao
    \item \num{5} łyżek gęstej śmietany
\end{Ingred}

Rozpuścić w~garnku margarynę, cukier puder i~kakao, dobrze wymieszać, gdy
przestygnie dodać śmietanę.

\recipe{Makowiec z~jabłkami}

\begin{Ingred}
    \item \qty{20}{\deca\gram} maku
    \item \num{1} szklanka cukru
    \item \num{.5} kostki margaryny
    \item \num{5} jaj
    \item \num{8} łyżek kaszy manny
    \item \qty{50}{\deca\gram} jabłek
    \item \num{1} łyżeczka proszku do pieczenia
\end{Ingred}

Jaja utrzeć z~cukrem, mak ugotować i~przekręcić \num{2} razy przez maszynkę,
dodać do ubitych jaj. Wlać roztopioną margarynę, wsypać kaszę, proszek i~jabłka
utarte na tarce jarzynowej. Wszystko starannie wymieszać, piec
\qtyrange{40}{45}{\minute}.

\recipe{Gofry}

\begin{Ingred}
    \item \qty{250}{\gram} mąki
    \item \qty{125}{\gram} masła
    \item \qty{50}{\gram} cukru
    \item \num{4} jajka
    \item \num{1} łyżeczka proszku do pieczenia
    \item \qty{.25}{\litre} śmietany lub mleka
    \item cukier waniliowy
    \item starta skórka z~cytryny
    \item szczypta soli
    \item olej
\end{Ingred}

Masło utrzeć z~cukrem i~cukrem waniliowym. Dodać sól oraz mąkę wymieszaną
z~proszkiem do pieczenia. Dodawać po jednym jajku. Ucierać ręcznie lub mikserem
do uzyskania gęstej, puszystej masy. Do dość gęstego ciasta dodać utartą skórkę
z~cytryny oraz śmietaną lub mleko. Starannie wymieszać.

\recipe{Porter}

\begin{Ingred}
    \item \num{1} szklanka spirytusu
    \item \num{1} szklanka cukru
    \item \num{2} cukry waniliowe
    \item \num{1} piwo
\end{Ingred}

Piwo, cukier oraz cukier waniliowy, zestawić z~ognia i~po chwili wlać spirytus.
Przykryć do wystygnięcia.

\recipe{Ciasto Pijak}

\begin{Ingred}
    \item \num{1} budyń śmietankowy
    \item \num{3} łyżki cukru pudru
    \item \num{6} białek
    \item \num{3} żółtka
    \item \qty{15}{\deka\gram} wiórków kokosowych
    \item \num{1} szklanka mleka
    \item \num{1} kostka masła lub margaryny \enquote{Kasia}
    \item \num{1} szklanka maku
    \item \num{2} paczki herbatników Petit
    \item niepełna szklanka cukru
    \item wódka
\end{Ingred}

Pianę ubić z~cukrem, dodać wiórki i~mak, dokładnie wymieszać. Wyłożyć do dużej
blachy i~włożyć do nagrzanego piekarnika na \qty{15}{\minute}.

Ugotować budyń z~szklanką mleka. Masło utrzeć z~cukrem pudrem i~żółtkami.
Wystudzony budyń dodawać po łyżce i~miksować. Krem wyłożyć na letnie ciasto,
herbatniki namoczyć wódką i~ułożyć na kremie, polać polewą.

\recipe{Miodowiec}

\begin{Ingred}[ciasto]
    \item \num{.5} szklanki cukru pudru
    \item \num{2} jajka
    \item \qty{.5}{\kilo\gram} mąki
    \item \num{.5} słoika sztucznego miodu
    \item \num{.5} kostki margaryny
    \item \num{1} łyżeczka sody
    \item \num{1} łyżeczka proszku do pieczenia
\end{Ingred}

\begin{Ingred}[masa]
    \item \qty{.5}{\litre} mleka
    \item \num{6} łyżek kaszy manny
    \item \num{1} kostka margaryny
    \item \num{1} szklanka cukru pudru
    \item sok z~cytryny
\end{Ingred}

Zagnieść ciasto. Gotować mleko z~kaszą manną. Utrzeć margarynę z~cukrem pudrem,
dodać zimną kaszę po łyżce. Na koniec dodać sok z~cytryny. Na wierzch
posmarować masą i~posypać okruchami ciasta.

\recipe{Wafle}

\begin{Ingred}
    \item \num{1} kostka margaryny
    \item \num{1} szklanka cukru
    \item \num{1} szklanka kwaśnej śmietany
    \item wafle
    \item galaretka poziomkowa lub morelowa
    \item \num{.75} szklanki mleka w~proszku
\end{Ingred}

Rozpuścić margarynę z~cukrem, dodać śmietanę cały czas mieszając. Gotować
\qty{5}{\minute} od momentu zagotowania. W gorącą masę wsypać galaretkę.
Wystudzić i~dodać mleko w~proszku.

\recipe{Pani Walewska}

\begin{Ingred}[ciasto]
    \item \num{3} szklanki mąki
    \item \num{1} kostka margaryny
    \item \num{1} szklanka cukru pudru
    \item \num{2} cukry waniliowe
    \item \num{2} żółtek
    \item \num{1.5} łyżeczki proszku do pieczenia
\end{Ingred}

\begin{Ingred}[krem]
    \item \num{1.5} szklanki mleka
    \item \num{6} łyżek cukru
    \item \num{2} cukry waniliowe
    \item \num{3} łyżki mąki ziemniaczanej
    \item \num{3} łyżki mąki pszennej
\end{Ingred}

Zagnieść ciasto, rozłożyć na \num{2} jednakowe blachy. Na jeden placek położyć
dżem, następnie pianę z~\num{5} białek, \num{.75} szklanki cukru pudru, \num{2}
łyżek mąki ziemniaczanej. Na pianę wyłożyć drobno pokrojone orzechy włoskie.
Piec około \qty{30}{\minute}. Drugi placek upiec, po czym posypać rodzynkami,
wyłożyć kremem i~przykryć pierwszym ciastem. By zrobić krem należy jego
składniki zagotować cały czas mieszając, następnie ostudzić i~dodać masło.

\recipe{Śmietanowiec}

\begin{Ingred}
    \item \qty{.5}{\litre} śmietany
    \item \num{1} szklanka cukru
    \item \num{3} płaskie łyżki żelatyny
    \item \num{.5} szklanki przegotowanej o~ostudzonej wody
    \item \num{2} galaretki
\end{Ingred}

Żelatynę namoczyć, następnie rozpuścić i~ostudzić. Śmietanę i~cukier lekko
zmiksować. Wlać żelatynę i~zmiksować, wylać do tortownicy. Rozrobić galaretki,
każdą oddzielnie i~zamrozić. Galaretkę pokroić w~kostkę i~włożyć do śmietany.
Wstawić do lodówki.

\recipe{Rolada waflowa}

\begin{Ingred}
    \item \num{5} tafli wafli
    \item \num{2} paczki herbatników
    \item \num{1} szklanka cukru
    \item \num{4} łyżki kakao
    \item \qty{10}{\deca\gram} wiórków kokosowych
    \item \qty{10}{\deca\gram} rodzynek
    \item \num{1.5} kostki masła
    \item \num{.75} szklanki cukru pudru
    \item \num{10} łyżek mleka
    \item galaretka
\end{Ingred}

Przygotować galaretkę dodając trochę mniej wody, niż zaleca przepis na
opakowaniu i~pozostawić do stężenia. Na masę kokosową utrzeć miękkie masło
z~cukrem pudrem. Zagotować mleko i~gorącym zalać wiórki kokosowe. Dodać
rodzynki i~wymieszać. Po przestudzeniu dodać do masy maślanej. Na masę kakaową
zagotować \num{1.5} szklanki wody z~cukrem i~\num{4} łyżkami kakao. Zdjąć
z~ognia, dodać pokruszone herbatniki i~wymieszać. Ciepłą masą kakaową
posmarować wafle. Na niej rozsmarować masę kokosową. Wafle namiękną i~staną się
wtedy giętkie. Stężałą galaretkę pokroić w~paski, ułożyć je wzdłuż boku każdego
wafla. Zwinąć je w~rulony. Roladę pokroić w~paski.

\recipe{Ciasto z~orzechami}

\begin{Ingred}
    \item \num{5} kwaśnych jabłek
    \item \num{1} szklanka cukru
    \item \num{3} jajka
    \item \num{1} szklanka zmiażdżonych orzechów
    \item \num{1} łyżeczka sody
    \item \num{1} łyżeczka proszku do pieczenia
    \item \num{2} szklanki mąki
    \item cukier waniliowy
\end{Ingred}

Jabłka zetrzeć, zasypać szklanką cukru, dodać resztę składników i~zagnieść
ciasto. Piec \qty{50}{\minute}.

\recipe{Kołacz}

\begin{Ingred}[ciasto \rom{1}]
    \item \num{4} jajka
    \item \num{.75} szklanki cukru
    \item \num{1} szklanka mąki
    \item \num{3.5} łyżki wody
    \item \num{5.5} łyżki oliwy
    \item \num{2} łyżeczki proszku do pieczenia
    \item cukier waniliowy
\end{Ingred}

\begin{Ingred}[ciasto \rom{2}]
    \item \num{4} jajka
    \item \num{.75} szklanki cukru
    \item \num{1} szklanka mąki
    \item \num{3.5} łyżki wody
    \item \num{5.5} łyżki oliwy
    \item \num{2} łyżeczki proszku do pieczenia
    \item \num{2} łyżki kakao
    \item cukier waniliowy
\end{Ingred}

\begin{Ingred}[masa]
    \item \num{1} kostka masła
    \item \qty{.75}{\litre} mleka
    \item cukier waniliowy lub zapach
    \item \num{8} łyżek cukru
    \item \num{2} łyżki mąki pszennej
    \item \num{3} łyżki mąki ziemniaczanej
\end{Ingred}

Aby przygotować ciasta, należy zmiksować jajka, cukier oraz cukier waniliowy,
dodając pozostałe składniki z~poszczególnych ciast. Na masę należy zagotować
i~wystudzić szklankę mleka, cukier, mąkę pszenną i~ziemniaczaną oraz cukier
waniliowy. Masło utrzeć i~dodać wystudzony budyń. Placki przekroić i~wyłożyć
budyniem.

\recipe{Ciasto do sernika}

\begin{Ingred}
    \item \qty{.75}{\kilo\gram} mąki
    \item \num{2} jajka
    \item \num{3} łyżki kakao
    \item \num{1} szklanka cukru
    \item \num{1.5} paczki proszku do pieczenia
\end{Ingred}

Margarynę rozpuścić. Ciasto zagnieść, włożyć do zamrażalnika. Dobrze zamrożone
ciasto utrzeć na dużych oczkach tarki, na spód sera oraz wierzch.

\recipe{Kremówka}

\begin{Ingred}[ciasto]
    \item \num{6} żółtek
    \item \num{6} łyżek cukru pudru
    \item \num{6} łyżek kwaśniej śmietany
    \item \num{3} płaskie łyżeczki proszku do pieczenia
    \item \num{2.5} szklanki mąki
\end{Ingred}

\begin{Ingred}[masa]
    \item \num{1.5} kostki masła
    \item \num{6} czubatych łyżek mąki
    \item \num{6} łyżek cukru
    \item \num{2} cukry waniliowe
    \item \num{3.75} szklanki mleka
\end{Ingred}

Ciasto zagnieść, podzielić na trzy części, wałkować i~upiec w~gorącym
piekarniku. By przygotować masę należy utrzeć masło, mąkę, cukier i~cukier
waniliowy. Mleko zagrzać, odlać szklankę, dodać do masy. Resztę mleka zagotować
i~wlać w~masę cały czas mieszając, aż do zgęstnięcia. Placki przekładać ciepłą
masą. Polać polewą.

\recipe{Sernik}

\begin{Ingred}[ciasto]
    \item \num{.5} szklanki cukru pudry
    \item \num{1.5} szklanki mąki
    \item \num{2} łyżeczki mąki
    \item \num{2} łyżeczki proszku do pieczenia
    \item \num{2} żółtka
    \item \num{2} łyżki kakao
    \item \qty{15}{\deca\gram} margaryny
    \item zapach rumowy
\end{Ingred}

\begin{Ingred}[ser]
    \item \qty{.5}{\kilo\gram} sera tłustego
    \item \num{5} jajek
    \item \num{1.5} kostki masła
    \item \num{1.5} szklanki cukru pudru
    \item \num{1} łyżka kaszy manny
    \item \num{1} łyżka mąki ziemniaczanej
    \item zapach pomarańczowy
\end{Ingred}

By przygotować ciasto, należy rozpuścić margarynę, ciasto zagnieść i~włożyć na
\qty{1}{\hour} do zamrażalnika. Przed włożeniem do zamrażalnika podzielić na
dwie nierówne części. Większa będzie na spód.

Ser i~masło zmielić w~maszynce. Żółtka, cukier zmiksować i~dodać do sera.
Następnie dodać mąkę i~kaszę, wymieszać. Na samym końcu ostrożnie wymieszać ser
z ubitymi na pianę białkami. Większą część ciasta zetrzeć na tarce, na to
wyłożyć ser i~na wierzch zetrzeć drugą część ciasta.

\recipe{Truskawki z~mascarpone}

\begin{Ingred}
    \item \num{2} żółtka
    \item \num{2} łyżki cukru
    \item \num{1} opakowanie serka mascarpone
    \item \qty{450}{\gram} truskawek
    \item \qty{30}{\gram} czekolady
\end{Ingred}

Żółtka dokładnie zmiksować z~cukrem. Gdy kogel nabierze jasnożółtej barwy,
dodawać do niego, ciągle mieszając serek mascarpone. Na dno niewielkich
pucharków nałożyć truskawki i~przykryć masą serową. Wstawić do lodówki. Przed
podaniem posypać startą czekoladą.

\recipe{Szarlotka Krysi}

\begin{Ingred}[biszkopt]
    \item \num{5} jaj
    \item \num{1} łyżeczka proszku do pieczenia
    \item \num{6} łyżek mąki
    \item \num{3} łyżki cukru
\end{Ingred}

\begin{Ingred}[masa jabłkowa]
    \item \qty{1.5}{\kilo\gram} jabłek
    \item \num{1} galaretka owocowa
    \item \num{1} łyżka żelatyny
    \item cukier
\end{Ingred}

\begin{Ingred}[krem]
    \item \qty{.5}{\litre} mleka
    \item \num{1} budyń cytrynowy
    \item \num{1} kostka masła
    \item \num{1} szklanka cukru
\end{Ingred}

\begin{Ingred}[polewa]
    \item \num{.5} kostki margaryny
    \item \num{4} łyżki cukru
    \item \num{1} łyżka wody
    \item \num{2} łyżki kakao
\end{Ingred}

\paru{Biszkopt} Białka ubić na sztywną pianę, dodać cukier, potem żółtka
i~delikatnie mieszając, wsypać mąkę wymieszaną z~proszkiem. Na tortownicę
wysmarowaną masłem i~wysypaną tartą bułką, należy wylać ciasto i~piec około
\qty{10}{\minute}.

\paru{Masa jabłkowa} Obrane i~pokrojone jabłka, bez gniazd, przesmażyć, wsypać
cukier i~galaretkę z~żelatyną. Chwilę pogotować. Na przestudzone ciasto wyłożyć
jabłka.

\paru{Krem} Ugotować budyń. Masło utrzeć z~cukrem na gładką masę i~dalej ucierając
dodawać po łyżce zimnego budyniu. Gotowy krem wyłożyć na warstwę jabłek.

\paru{Polewa} Składniki polewy podgrzać do zagotowania, ale nie gotować. Ciągle
mieszając doprowadzić do ostudzenia. Oblać ciasto zimną polewą.

\recipe{Ciasteczka od Ewy}

\begin{Ingred}
    \item \num{1} kostka masła ekstra o~temperaturze pokojowej
    \item \num{1.5} szklanki cukru
    \item \num{2} jaja
    \item \num{2} szklanki mąki
    \item \num{1} łyżeczka sody
    \item \num{1} łyżeczka cynamonu
    \item \num{.5} łyżeczki soli
    \item \num{1} szklanka rodzynek (około \qty{20}{\deka\gram})
    \item \num{3} szklanki płatków owsianych błyskawicznych
    \item cukier waniliowy
\end{Ingred}

Masło rozetrzeć z~cukrem i~cukrem waniliowym, dodać jaja, następnie mąkę, sodę
i~sól, dokładnie wymieszać. Na koniec dodać rodzynki i~płatki owsiane. Kłaść
łyżką na blachę i~trochę spłaszczyć. Piec około \qtyrange{10}{15}{\minute}.

\recipe{Fuga}

\begin{Ingred}
    \item \num{8} białek
    \item \num{4.5} szklanki mąki
    \item \num{3} łyżeczki proszku do pieczenia
    \item \num{2} szklanki cukru
    \item \num{1.5} kostki margaryny
    \item \num{1} cytryna
    \item rodzynki
    \item orzechy
    \item kakao
    \item marmolada
\end{Ingred}

Odstawić jedną szklankę cukru do piany. Margarynę posiekać z~mąką, cukrem
proszkiem do pieczenia. Dodać sok z~cytryny. Wyrobić ciasto, podzielić na trzy
części, z~tego jedną większą. Do jednej z~mniejszych części dodać kakao. Dwie
mniejsze porcje włożyć do lodówki lub zamrażalki. Dużą część rozłożyć na
blaszce, posmarować marmoladą, nasypać na to rodzynki, orzechy i~utarte ciemne
ciasto. Z ośmiu białek ubić pianę, na koniec ubijania dodać szklankę cukru.
Pianę rozłożyć na ciemne ciasto i~na wierzch posypać startym jasnym ciastem.
Piec około \qty{55}{\minute}.

\recipe{Pleśniak}

\begin{Ingred}
    \item \qty{.5}{\kilo\gram} mąki
    \item \num{1} szklanka cukru
    \item \num{5} jajek
    \item \num{1} kostka masła lub margaryny
    \item \num{1} łyżeczka proszku do pieczenia
    \item marmolada
\end{Ingred}

Wszystkie białka utrzeć z~\num{.5} szklanki cukru. Resztę składników utrzeć,
podzielić na trzy części, jedną wymieszać z~kakao. Zetrzeć na tarce białe
ciasto, następnie ciemne. Ciemne ciasto wysmarować marmoladą, na wierzch dodać
utarte białka. Na samą górę zetrzeć ostatnią część jasnego ciasta. Piec około
\qty{45}{\minute}.

\recipe{Sernik z~brzoskwinią}

\begin{Ingred}[ciasto]
    \item \num{3} szklanki mąki
    \item \num{1} margaryna \enquote{Kasia}
    \item \num{3} łyżki cukru
    \item \num{5} żółtek
    \item \num{3} łyżeczki proszku do pieczenia
\end{Ingred}

\begin{Ingred}[masa serowa]
    \item \qty{1}{\kilo\gram} sera śmietankowego
    \item \num{2} jaja
    \item \num{1} margaryna \enquote{Kasia}
    \item \num{1} budyń śmietankowy
    \item \num{1} szklanka cukru
    \item \num{1} puszka brzoskwiń
\end{Ingred}

\paru{Ciasto} Ciasto podzielić na dwie części. Jedną część wyłożyć do brytfanki
i~wstawić do lodówki na \qty{2}{\hour}. Drugą część włożyć do zamrażalnika na
\qty{2}{\hour}.

\paru{Masa serowa} Wszystkie składniki, oprócz brzoskwiń, zmiksować. Wylać na
ciasto do brytfanki. Na tą masę ułożyć pokrojone brzoskwinie. Z pięciu białek
ubić na sztywno pianę z~\num{.75} szklanki cukru i~wyłożyć na pokrojone
brzoskwinie. Ciasto z~zamrażalnika utrzeć na tarce z~dużymi oczkami. Wysypać je
na pianę. Piec około \qty{1}{\hour} w~temperaturze \qty{220}{\celsius}.

\recipe{Cycki murzynki}

\begin{Ingred}[ciasto]
    \item \num{9} białek
    \item \num{9} łyżek cukru
    \item \num{1.5} szklanki cukru
    \item \num{1.5} szklanki wiórek kokosowych
\end{Ingred}

\begin{Ingred}[masa]
    \item \num{3} szklanki mleka
    \item \num{3} łyżki mąki pszennej
    \item \num{3} łyżki mąki ziemniaczanej
    \item \num{1} szklanka cukru
    \item \num{4} żółtka
    \item \num{1.5} kostki masła lub margaryny
    \item cukier waniliowy
\end{Ingred}

\begin{Ingred}[polewa]
    \item \num{.5} kostki masła
    \item \num{2} łyżki mleka
    \item \num{.5} szklanki cukru
    \item \num{2} łyżki kakao
\end{Ingred}

\paru{Ciasto} Białko ubić, utrwalić cukrem, dodać mak i~wiórki. Masę wyłożyć do
formy wysmarowanej masłem i~wysypanej bułką.

\paru{Masa} Ugotować budyń, ostudzić. Masło utrzeć i~dodać ostudzony budyń. Na
ostudzone ciasto wyłożyć masę. Na wierzchu układać biszkopty moczone w~koniaku
lub innym alkoholu. Ciasto polać polewą.

\recipe{Ciasto dyniowe \enquote{Mokre}}

\begin{Ingred}
    \item \num{2} szklanki dyni startej na jarzynowej tarce
    \item \num{1} szklanka curry
    \item \num{3} jajka
    \item \num{3} łyżeczki cynamonu
    \item \num{1} łyżeczka sody
    \item \num{3} szklanki mąki
    \item \num{1} szklanka oleju
    \item cukier waniliowy
    \item garść włoskich
\end{Ingred}

Jajka ubić z~cukrem, dodać olej, mąkę, cukier waniliowy, sodę i~cynamon.
Zmiksować, do masy dodać utartą dynię oraz orzechy, delikatnie wymieszać łyżką.
Ciasto wlać do formy wysmarowanej masłem i~wysypanej mąką. Piec w~piekarniku
nagrzanym do \qty{180}{\celsius} przez \qty{50}{\minute}. Po upieczeniu
ostudzone ciasto posypać cukrem pudrem.

\recipe{Kokosowiec}

\begin{Ingred}[biszkopt]
    \item \num{4} jajka
    \item \num{4} łyżki mąki
    \item \num{4} łyżki cukru
    \item \num{1} łyżeczka proszku do pieczenia
\end{Ingred}

\begin{Ingred}[ciasto wiórkowe]
    \item \num{4} białka
    \item \num{1} szklanka cukru
    \item \num{2} paczki wiórków kokosowych
\end{Ingred}

\begin{Ingred}[krem]
    \item \num{1} budyń śmietankowy
    \item \num{1} łyżeczka cukru
    \item \num{1} łyżeczka mąki
    \item \num{4} żółtka
    \item sok z~jednej puszki ananasów
\end{Ingred}

\paru{Biszkopt} Białka ubić na pianę, dodać cukier i~żółtka, następnie dodać mąką
wymieszaną z~proszkiem do pieczenia. Delikatnie wymieszać i~upiec. Na tej samej
blaszce upiec ciasto wiórkowe.

\paru{Ciasto wiórkowe} Białka ubić na pianę, dodać wiórki i~wymieszać. Upiec na
złoty kolor, około \qty{20}{\minute}.

\paru{Krem} Łyżeczkę cukru, mąki oraz budyń rozmieszać w~\num{.33} szklanki wody.
Ugotować budyń z~soku ananasów i~wystudzić. Ananasy pokroić w~drobną kostkę.
Kostkę margaryny zmiksować, dodać \num{2} żółtka i~po trochu ugotowany budyń.
Zmiksować na puszystą masę, dodać ananasy i~wymieszać. Na biszkopt wyłożyć
krem, a~na krem położyć ciasto wiórkowe.

\recipe{Makowe}

\begin{Ingred}
    \item \num{.5} torebki maku
    \item \num{10} jajek
    \item \qty{40}{\deka\gram} cukru pudru
    \item zapach migdałowy
\end{Ingred}

Mak sparzyć na noc, zmielić trzy razy. Żółtka utrzeć z~cukrem na pulchną masę,
białka ubić, dodać wszystko do maku. Wymieszać i~piec.

\recipe{Styropian}

\begin{Ingred}
    \item \num{3} szklanki mąki
    \item \num{.75} szklanki cukru
    \item \num{1} paczka margaryny
    \item \num{1} łyżeczka proszku do pieczenia
    \item \num{5} żółtek
\end{Ingred}

\recipe{Ciasto dyniowe}

\begin{Ingred}
    \item \qty{250}{\gram} mąki krupczatki
    \item \num{3} łyżki cukru pudru
    \item \qty{250}{\gram} masła
    \item \num{2} żółtka
    \item \qty{40}{\gram} orzechów włoskich
    \item \qty{340}{\gram} miąższu dyni
    \item \num{2} jajka
    \item \qty{70}{\gram} brązowego cukru
    \item \qty{275}{\milli\litre} śmietany \qty{36}{\percent}
    \item \num{1} łyżeczka mielonego cynamonu
    \item szczypta imbiru
    \item szczypta gałki muszkatołowej
    \item szczypta goździków
    \item szczypta soli
    \item cukier waniliowy
\end{Ingred}

Mąkę, cukier, masło, żółtko, cukier waniliowy i~szczyptę soli wymieszać
i~posiekać nożem. Zagnieść, dodając na końcu rozdrobnione orzechy. Włożyć do
lodówki na \qty{1}{\hour}. Wyłożyć ciasto na tortownicę o~średnicy
\qty{23}{\centi\metre}. Ugotować dynię w~małej ilości wody, odsączyć, Utrzeć
jajka z~żółtkiem i~cukrem. Do rondelka wlać śmietanę, dodać przyprawy
i~zagotować. Połączyć z~utartymi jabłkami i~pure z~dyni, wymieszać. Masę wylać
na ciasto i~piec \qty{35}{\minute} w~temperaturze \qty{180}{\celsius}. Ostudzić
i~posypać cukrem.

\end{document}
