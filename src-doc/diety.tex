\documentclass[../kucharek.tex]{subfiles}
\begin{document}

\recipe{Owsianka z~chia -- śniadanie}

\begin{Ingred}
    \item \qty{6}{\gram} wiórków kokosowych
    \item \qty{50}{\gram} borówki amerykańskiej
    \item \qty{100}{\gram} malin
    \item \qty{20}{\gram} płatków owsianych
    \item \qty{20}{\gram} suszonych nasion chia
    \item \qty{250}{\milli\litre} mleka owsianego
\end{Ingred}

Płatki zalać mlekiem, dodać chia, zagotować. Posypać malinami i~borówką, a~na
koniec wiórkami.

\recipe{Sałatka z~szarpanym kurczakiem -- \rom{2} śniadanie}

\begin{Ingred}
    \item \qty{100}{\gram} mięso z~piersi kurczaka bez skóry
    \item \qty{35}{\gram} chleba żytniego
    \item \qty{80}{\gram} kapusty białej
    \item \qty{100}{\gram} ogórka
    \item \qty{10}{\milli\litre} oliwy z~oliwek
    \item \qty{1}{\gram} suszonej bazylii
    \item \qty{1}{\gram} suszonego oregano
    \item \qty{1}{\gram} pieprzu czarnego
    \item \qty{60}{\gram} rzodkiewki
    \item \qty{5}{\gram} soli
    \item \qty{10}{\gram} pestek dyni
\end{Ingred}

Piersi z~kurczaka umyj i~osusz. Przypraw solą, pieprzem, zgrilluj. W tym czasie
pokrój warzywa -- rzodkiewki oraz ogórki na plasterki, a~kapustę cienko
poszatkuj. Podpraż na patelni nasiona. Mięso wyjmij na talerz i~przy pomocy
dwóch widelców rozrywaj je wzdłuż włókien. Połącz z~pozostałymi składnikami
i~polej dressingiem z~oliwy z~oliwek i~ziół.

Chleb pokroić na małe kawałki. Przyprawić i~podsmażyć na suchej patelni.
Zamiast kapusty można użyć dowolnego miksu sałat.

\recipe{Paella z~krewetkami -- obiad}

\begin{Ingred}
    \item \qty{25}{\gram} cebuli
    \item \qty{2}{\gram} czosnku
    \item \qty{70}{\gram} mrożonego groszku zielonego
    \item \qty{50}{\gram} koncentratu pomidorowego
    \item \qty{5}{\milli\litre} oleju kokosowego
    \item \qty{114}{\gram} czerwonej papryki
    \item \qty{12}{\gram} liści pietruszki
    \item \qty{1}{\gram} papryki w~proszku
    \item \qty{50}{\gram} białego ryżu
    \item \qty{200}{\milli\litre} bulionu rosołowego
    \item \qty{100}{\gram} surowych krewetek
\end{Ingred}

Na patelni rozgrzać olej, dodać pokrojoną w~kosteczkę cebulę i~zeszklić ją.
Dodać starty czosnek i~pokrojoną w~kosteczkę paprykę. Smażyć mieszając przez
około \qty{3}{\minute}. Dorzucić paprykę w~proszku, przecier pomidorowy a~po
chwili wsypać ryż i~mieszając smażyć \qty{2}{\minute}. Wlać zagotowany bulion,
dodać mrożony groszek i~przemieszać. Zagotować na dużym ogniu, następnie
gotować bez mieszania przez \qty{8}{\minute}. Wyłożyć na wierzch krewetki,
zmniejszyć ogień, zawinąć patelnie folią aluminiową i~gotować \qty{6}{\minute}.
Odstawić z~ognia i~trzymać pod przykryciem przez \qty{2}{\minute}. Podawać
z~posiekaną natką pietruszki

\recipe{Kanapki z~pastą rybną -- kolacja}

\begin{Ingred}
    \item \qty{10}{\gram} szczypiorku
    \item \qty{70}{\gram} chleba żytniego
    \item \qty{60}{\gram} wędzonej makreli
    \item \qty{10}{\gram} musztardy
    \item \qty{100}{\gram} ogórka
    \item \qty{28}{\gram} pomidorów suszonych na słońcu
    \item \qty{1}{\gram} pieprzu czarnego
    \item \qty{1}{\gram} soli białej
    \item \qty{15}{\gram} marynowanych oliwek zielonych
\end{Ingred}

Rybę oczyścić z~ości i~skóry, przełożyć do blendera i~zmiksować z~oliwkami,
musztardą, pomidorami i~szczypiorkiem. Doprawić do smaku. Gotową pastę
posmarować pieczywo, ułożyć ogórek.

\recipe{Nocna owsianka o~smaku orzechowym -- śniadanie}

\begin{Ingred}
    \item \qty{30}{\gram} płatków owsianych
    \item \qty{1}{\gram} soli
    \item \qty{100}{\milli\litre} mleka \qty{1.5}{\percent}
    \item \qty{6}{\gram} orzeszków ziemnych
    \item \qty{160}{\milli\litre} jogurtu naturalnego (Skyr)
    \item \qty{20}{\gram} masła orzechowego z~kawałkami orzechów
    \item \qty{7}{\gram} ksylitolu
\end{Ingred}

Do miseczki lub do słoika wsypać płatki owsiane i~erytrytol lub inny słodzik
bez kalorii. Wymieszać. Dodać masło orzechowe i~szczyptę soli. Dolać mleko i~w
razie potrzeby trochę wody mineralnej (ilość dostosować tak, aby przykryć
zawartość naczynia). Dokładnie mieszać do momentu, aż wszystkie składniki będą
dobrze połączone. Naczynie przykryć folią lub przykrywką i~wstawić do lodówki
na kilka godzin, a~najlepiej na całą noc. Płatki w~tym czasie napęcznieją,
a~smaki dobrze się połączą. Owsianka po wyciągnięciu z~lodówki jest gotowa do
zjedzenia. Drobno posiekaj orzechy Owsiankę zjedz na zimno lub po podgrzaniu
w~rondelku. Posyp górę owsianki orzechami. Wymieszać z~jogurtem.

\recipe{Bułka razowa z~hummusem -- \rom{2} śniadanie}

\begin{Ingred}
    \item \qty{90}{\gram} bułki grahamki
    \item \qty{30}{\gram} szynki z~piersi kurczaka
    \item \qty{25}{\gram} hummusu
    \item \qty{20}{\gram} sałaty masłowej
    \item \qty{60}{\gram} ogórka kiszonego
\end{Ingred}

\recipe{Słodko kwaśny kurczak z~mango -- obiad}

\begin{Ingred}
    \item \qty{100}{\gram} mięso z~piersi kurczaka, bez skóry
    \item \qty{80}{\gram} brokuła
    \item \qty{50}{\gram} cebuli
    \item \qty{3}{\gram} korzeń imbiru
    \item \qty{80}{\gram} mango
    \item \qty{12}{\gram} miód pszczeli
    \item \qty{3}{\milli\litre} ocet balsamiczny
    \item \qty{10}{\milli\litre} olej kokosowy
    \item \qty{1}{\gram} pieprz czarny
    \item \qty{50}{\gram} ryżu białego
    \item \qty{2}{\milli\litre} sos sojowy
    \item \qty{1}{\gram} sól biała
    \item \qty{10}{\milli\litre} wino białe, wytrawne
    \item \qty{15}{\gram} orzeszki ziemne
\end{Ingred}

Odrobinę wody, wino, olej, ocet, sos sojowy i~miód połączyć. Przekroić mango na
pół i~wyjąć pestkę. Obrać owoc, a~następnie pokroić w~kostkę. Podgrzać połowę
oleju w~woku lub dużej patelni, dodać filety z~kurczaka i~smażyć na dużym ogniu
(każda ze stron po \qtyrange{2}{3}{\minute}). Przełożyć na talerz. Podgrzać
resztę oleju, dodać cebulę i~smażyć \qtyrange{1}{2}{\minute}, aż zmięknie.
Dodać brokuły i~imbir, a~potem sos i~mango. Doprowadzić do wrzenia i~dusić na
wolnym ogniu przez około \qty{3}{\minute}. Przełożyć kurczaka do woka i~nadal
dusić około \qtyrange{2}{3}{\minute}, aż będzie miękki. Podawać z~ryżem
i~prażonymi orzeszkami ziemnymi.

\recipe{Sałatka z~ryżem i~krewetkami -- kolacja}

\begin{Ingred}
    \item \qty{10}{\gram} musztardy
    \item \qty{90}{\gram} ogórka
    \item \qty{10}{\milli\litre} oliwy z~oliwek
    \item \qty{90}{\gram} pomidora
    \item \qty{1}{\gram} papryki w~proszku
    \item \qty{30}{\gram} ryżu
    \item \qty{45}{\gram} rzodkiewki
    \item \qty{40}{\gram} sałaty lodowej
    \item \qty{100}{\gram} krewetki surowe
    \item \qty{10}{\gram} szczypiorku
\end{Ingred}

Ryż ugotować do miękkości w~osolonej wodzie. Pozostawić do wystygnięcia. Liście
sałaty pokroić w~paseczki, wrzucić do salaterki, dodać plasterki ogórka
i~kawałki pomidorów, zmieszać i~odsunąć na brzegi, pozostawiając środek
salaterki na ryż. W innym naczyniu zmieszać ryż z~grillowanymi krewetkami,
dodać dressing (oliwa z~oliwek i~musztarda). Przełożyć na środek salaterki
z~warzywami.

\recipe{Tosty z~pieczarkami -- śniadanie}

\begin{Ingred}
    \item \qty{20}{\gram} cebuli
    \item \qty{70}{\gram} chleba żytniego
    \item \qty{50}{\gram} papryki czerwonej
    \item \qty{100}{\gram} pieczarek
    \item \qty{1}{\gram} pieprzu czarnego
    \item \qty{30}{\gram} sera goudy
\end{Ingred}

Pieczarki umyć, zetrzeć na tarce o~grubych oczkach, poddusić w~rondelku
z~cebulką. Na chleb razowy ułożyć plastry sera, podduszone pieczarki, paprykę
oraz kukurydzę. Można doprawić do smaku. Wyłożyć do tostera.

\recipe{Jogurt z~grejpfrutem -- \rom{2} śniadanie}

\begin{Ingred}
    \item \qty{10}{\gram} gorzkiej czekolady
    \item \qty{120}{\gram} grejpfruta
    \item \qty{30}{\gram} płatków owsianych
    \item \qty{180}{\gram} jogurtu naturalnego \qty{2}{\percent}
\end{Ingred}

Obrać grejpfruta, pokroić w~kostkę i~wymieszać z~jogurtem. Czekoladę zetrzeć na
tarce i~posypać nią jogurt.

\recipe{Czosnkowo-paprykowe udka z~kurczaka z~frytkami -- obiad}

\begin{Ingred}
    \item \qty{12}{\gram} czosnku
    \item \qty{100}{\gram} frytek bez soli
    \item \qty{10}{\milli\litre} oliwy z~oliwek
    \item \qty{18}{\gram} natki pietruszki
    \item \qty{1}{\gram} suszonego oregano
    \item \qty{3}{\gram} papryki w~proszku
    \item \qty{1}{\gram} pieprzu czarny
    \item \qty{1}{\gram} soli
    \item \qty{150}{\gram} mięsa z~udka kurczaka, bez skóry
\end{Ingred}

Rozgrzać piekarnik do \qty{200}{\celsius}. Oczyścić i~wysuszyć udka i~doprawić
solą i~pieprzem. Odłożyć na bok. W małym rondelku rozgrzać oliwę z~oliwek.
Dodać czosnek, paprykę, pietruszkę i~oregano. Podgrzewać przez około
\qty{1}{\minute} na średnim ogniu, nie przypalić ziół i~czosnku. Wlać
mieszaninę oleju i~ziół na udka i~upewnić się, że są dokładnie nią pokryte.
Umieścić udka w~naczyniu żaroodpornym i~piec przez około \qty{45}{\minute} lub
do momentu, kiedy udka będą ugotowane. Pod koniec pieczenia włączyć termoobieg
żeby udka się bardziej przypiekły. Frytki wrzucić około
\qtyrange{20}{25}{\minute} przed końcem.

\recipe{Sałatka z~pomidora, cebuli i~jogurtu naturalnego}

\begin{Ingred}
    \item \qty{330}{\gram} pomidorów
    \item \qty{1}{\gram} soli
    \item \qty{30}{\gram} cebuli dymki
    \item \qty{90}{\gram} jogurtu naturalnego \qty{2}{\percent}
\end{Ingred}

Należy pokroić pomidora i~cebulę. Następnie wsypać warzywa do miski i~wymieszać
z~jogurtem naturalnym. Dodać szczyptę soli.

\recipe{Tortilla z~guacamole na ostro z~kurczakiem i~sosem czosnkowym -- kolacja}

\begin{Ingred}
    \item \qty{40}{\gram} awokado
    \item \qty{50}{\gram} cebuli
    \item \qty{10}{\gram} czosnku
    \item \qty{80}{\gram} mięsa z~piersi indyka, bez skóry
    \item \qty{2}{\gram} papryki chili
    \item \qty{100}{\gram} sałaty lodowej
    \item \qty{15}{\gram} sera goudy
    \item \qty{30}{\gram} sałaty zielonej
    \item \qty{80}{\milli\litre} jogurtu naturalnego Skyr
    \item \qty{60}{\gram} tortilli pełnoziarnistej
    \item \qty{40}{\gram} konserwowej kukurydzy
\end{Ingred}

Zrobić dip z~awokado i~soku limonki oraz czosnku. Posmarować tortille, dodać
dowolną sałatę, cebulę, kukurydzę i~papryczkę chilli. Tortille grillować lub
wcześniej placek podgrzać w~mikrofalówce i~nakładać dodatki. Podawać
z~kurczakiem wędzonym lub grillowanym. Jogurt Skyr wymieszać z~czosnkiem
i~przyprawami, stosować jako sos.

\recipe{Kanapka z~jajkiem -- śniadanie}

\begin{Ingred}
    \item \qty{100}{\gram} jajek
    \item \qty{30}{\gram} cebuli
    \item \qty{70}{\gram} chleba żytniego
    \item \qty{60}{\gram} ogórka
    \item \qty{1}{\gram} suszonej bazylii
    \item \qty{1}{\gram} suszonego majeranku
    \item \qty{30}{\gram} rzodkiewki
    \item \qty{30}{\gram} szynki z~piersi kurczaka
\end{Ingred}

Ogórka i~cebulę pokroić w~kostkę i~wymieszać dodając majeranek i~bazylię. Na
kanapce ułożyć pokrojone w~plasterki jajko i~warzywa.

\recipe{Leczo z~fasolką szparagową -- obiad}

\begin{Ingred}
    \item \qty{110}{\gram} cebuli
    \item \qty{240}{\gram} cukinii
    \item \qty{10}{\milli\litre} oliwy z~oliwek
    \item \qty{240}{\gram} papryki czerwonej
    \item \qty{330}{\gram} pomidorów
    \item \qty{1}{\gram} papryki chili w~proszku
    \item \qty{1}{\gram} suszonego majeranku
    \item \qty{1}{\gram} pieprzu czarnego
    \item \qty{1}{\gram} soli
    \item \qty{100}{\gram} fasolki szparagowej
    \item \qty{100}{\gram} kiełbasy z~kurczaka
\end{Ingred}

Pokrój paprykę, cukinię oraz cebulę. Zalej niewielką ilością wrzątku i~dodaj
oliwę. Duś pod przykryciem. Dodaj pokrojoną na mniejsze kawałki fasolkę oraz
pomidora. Przypraw pieprzem, ziołami prowansalskimi, solą i~chili.

\recipe{Sałatka z~gruszki i~buraka -- kolacja}

\begin{Ingred}
    \item \qty{100}{\gram} buraków
    \item \qty{120}{\gram} gruszki
    \item \qty{10}{\milli\litre} oliwy z~oliwek
    \item \qty{1}{\gram} pieprzu czarnego
    \item \qty{40}{\gram} rukola
    \item \qty{40}{\gram} sera brie
    \item \qty{1}{\gram} soli
    \item \qty{12}{\milli\litre} soku z~cytryny
    \item \qty{8}{\gram} orzechów włoskich
\end{Ingred}

Ugotowanego buraka pokroić w~kostkę. Obrać gruszkę i~pokroić również w~kostkę.
Wymieszać z~rukolą, serem brie. Polać oliwą z~oliwek, sokiem z~cytryny
i~posypać orzechami włoskimi. Doprawić do smaku

\recipe{Migdałowa granola -- śniadanie}

\begin{Ingred}
    \item \qty{100}{\gram} borówki amerykańskiej
    \item \qty{30}{\gram} płatków owsianych
    \item \qty{2}{\gram} mielonego cynamonu
    \item \qty{10}{\gram} migdałów
    \item \qty{180}{\gram} jogurtu naturalnego \qty{2}{\percent}
\end{Ingred}

Wsypać płatki na patelnię. Smażyć na niedużym ogniu, mieszając aż się płatki
lekko zarumienią. Zdjąć z~ognia, dodać cynamon i~płatki migdałów. Można dodać
cynamon. Wymieszać i~przełożyć na głęboki talerz do ostudzenia. Wymieszać
granolę z~jogurtem naturalnym.

\recipe{Orzechowy deser egzotyczny -- \rom{2} śniadanie}

\begin{Ingred}
    \item \qty{10}{\gram} łuskanych pestek dyni
    \item \qty{79}{\gram} kiwi
    \item \qty{60}{\gram} mandarynek
    \item \qty{12}{\gram} orzechów laskowych
    \item \qty{20}{\gram} suszonych nasion chia
    \item \qty{180}{\gram} jogurtu naturalnego \qty{2}{\percent}
\end{Ingred}

Mandarynkę obrać i~podzielić na kawałki. Orzechy posiekać, wymieszać z~pestkami
i~mandarynką.

\recipe{Makaron z~ciecierzycą, rukolą i~pesto -- obiad}

\begin{Ingred}
    \item \qty{15}{\gram} orzechów włoskich
    \item \qty{15}{\gram} czosnku
    \item \qty{15}{\gram} pomidorów koktajlowych
    \item \qty{15}{\gram} suszonego oregano
    \item \qty{15}{\gram} pieprzu czarnego
    \item \qty{15}{\gram} rukoli
    \item \qty{15}{\gram} soli
    \item \qty{15}{\gram} ciecierzycy z~puszki
    \item \qty{15}{\gram} sosu pesto
    \item \qty{15}{\gram} makaronu z~pszenicy durum
\end{Ingred}

Wstaw wodę na makaron, jak zacznie się gotować wrzuć spaghetti i~ugotuj
Al~Dente. Makaron wymieszaj z~pesto, dodaj rukolę, pomidorki koktajlowe,
ciecierzycę konserwową oraz orzechy. Można je wcześniej podprażyć. Dopraw do
smaku.

\recipe{Szakszuka z~szynką -- kolacja}

\begin{Ingred}
    \item \qty{100}{\gram} jajek
    \item \qty{3}{\gram} świeżej bazylii
    \item \qty{35}{\gram} chleba żytniego
    \item \qty{30}{\gram} koncentratu pomidorowego \qty{30}{\percent}
    \item \qty{10}{\milli\litre} oliwy z~oliwek
    \item \qty{1}{\gram} pieprzu czarnego
    \item \qty{20}{\gram} rukoli
    \item \qty{1}{\gram} soli
    \item \qty{15}{\gram} sera gouda
    \item \qty{1}{\gram} świeża kolendra
    \item \qty{30}{\gram} szynki parmeńskiej
    \item \qty{40}{\gram} kukurydzy konserwowej
    \item \qty{40}{\gram} konserwowa fasola czerwona
\end{Ingred}

Pokrojone pomidory podlane wodą podgrzać na oliwie z~dodatkiem koncentratu.
Można użyć gotowej passaty pomidorowej lub pomidorów z~puszki. Doprawić solą
pieprzem i~ziołami. Po podduszeniu dodać garść rukoli, kawałki szynki
i~wymieszać, dodać czerwoną fasolę, kukurydzę. Na koniec wbić jajka i~przykryć,
dusić, aż do ścięcia białek. Można posypać kolendrą albo natką pietruszki oraz
żółtym serem.

\recipe{Tosty francuskie z~szynką -- śniadanie}

\begin{Ingred}
    \item \qty{50}{\gram} jajek
    \item \qty{70}{\gram} chleba żytniego
    \item \qty{100}{\gram} pomidorów koktajlowych
    \item \qty{1}{\gram} suszonego oregano
    \item \qty{1}{\gram} soli
    \item \qty{10}{\gram} masła
    \item \qty{15}{\gram} szynki z~piersi kurczaka
    \item \qty{10}{\milli\litre} mleka \qty{1.5}{\percent}
    \item \qty{30}{\gram} sera goudy
\end{Ingred}

Jajka roztrzepujemy z~mlekiem, dodajemy oregano i~szczyptę soli. Na dwóch
kromkach chleba układamy ser i~szynkę, przykrywamy pozostałymi kromkami. Tak
przygotowane kanapki obtaczamy w~jajku. Na patelni rozgrzewamy masło, smażymy
kanapki z~obu stron, aż masa jajeczna ładnie się zrumieni. Podajemy od razu po
przygotowaniu. Zjeść z~pomidorkami koktajlowymi.

\recipe{Koktajl leśny}

\begin{Ingred}
    \item \qty{50}{\gram} borówki amerykańskiej
    \item \qty{120}{\gram} malin
    \item \qty{10}{\gram} otrębów pszennych
    \item \qty{50}{\milli\litre} mleka \qty{1.5}{\percent}
    \item \qty{10}{\gram} suszonych nasion chia
    \item \qty{180}{\gram} serka wiejskiego
    \item \qty{7}{\gram} ksylitolu
\end{Ingred}

Zmiksować wszystkie produkty na gładką masę.

\recipe{Naleśniki budyniowe z~twarogiem -- obiad}

\begin{Ingred}
    \item \qty{50}{\gram} jajek
    \item \qty{100}{\gram} borówki amerykańskiej
    \item \qty{20}{\gram} gorzkiej czekolady
    \item \qty{60}{\gram} jogurtu naturalnego \qty{2}{\percent}
    \item \qty{10}{\milli\litre} oliwy z~oliwek
    \item \qty{100}{\gram} sera twarogowego półtłustego
    \item \qty{90}{\milli\litre} mleka \qty{1.5}{\percent}
    \item \qty{40}{\gram} budyniu w~proszku
\end{Ingred}

Jajo, proszek budyniowy i~mleko roztrzepać na gładką masę i~usmażyć naleśniki.
Twaróg wymieszać z~jogurtem i~kawałkami czekolady. Opcjonalnie dosłodzić
erytrolem lub miodem. Faszerować naleśniki, na wierzchu ułożyć borówki, posypać
startą czekoladą.

\recipe{Pasta jajeczna z~tuńczykiem -- kolacja}
\begin{Ingred}
    \item \qty{50}{\gram} jajek
    \item \qty{70}{\gram} chleba żytniego
    \item \qty{20}{\gram} jogurtu naturalnego \qty{2}{\percent}
    \item \qty{1}{\gram} pieprzu czarnego
    \item \qty{1}{\gram} soli
    \item \qty{60}{\gram} tuńczyka w~sosie własnym
    \item \qty{60}{\gram} ogórków kiszonych
\end{Ingred}

Jajko ugotować, obrać, pokroić na kawałki i~włożyć do miseczki. Rozgnieść
widelcem najdrobniej jak się da. Do jajek wlać jogurt. Przyprawić pieprzem.
Wymieszać lub zblendować, aby składniki dobrze się połączyły. Tuńczyka
odsączyć, rozdrobnić i~dodać do masy. Dodać ogórka kiszonego.

\recipe{Twarożek ogórkowy -- śniadanie}

\begin{Ingred}
    \item \qty{70}{\gram} chleba żytniego
    \item \qty{20}{\gram} jogurtu naturalnego \qty{2}{\percent}
    \item \qty{10}{\gram} kopru ogrodowego
    \item \qty{50}{\gram} ogórka
    \item \qty{1}{\gram} pieprzu czarnego
    \item \qty{120}{\gram} sera twarogowego półtłustego
    \item \qty{1}{\gram} soli
\end{Ingred}

Ogórka pokroić w~kostkę, koperek posiekać. Ser twarogowy rozgnieć widelcem
i~wymieszać z~jogurtem naturalnym. Doprawić do smaku solą i~pieprzem. Dokładnie
wymieszać.

\recipe{Koktajl rafaello -- \rom{2} śniadanie}

\begin{Ingred}
    \item \qty{6}{\gram} wiórków kokosowych
    \item \qty{130}{\gram} banana
    \item \qty{40}{\gram} jogurtu naturalnego \qty{2}{\percent}
    \item \qty{15}{\gram} migdałów
    \item \qty{250}{\milli\litre} mleka \qty{1.5}{\percent}
\end{Ingred}

Zmiksować wszystko na gładką masę. Mleko można zastąpić na mleko owsianego lub
kokosowe.

\recipe{Bowl z~kaszą bulgur -- obiad}

\begin{Ingred}
    \item \qty{100}{\gram} brokułów
    \item \qty{50}{\gram} groszka zielonego
    \item \qty{20}{\gram} marchwi
    \item \qty{10}{\milli\litre} oliwy z~oliwek
    \item \qty{1}{\gram} suszonej bazylii
    \item \qty{1}{\gram} pieprzu czarnego
    \item \qty{1}{\gram} soli
    \item \qty{50}{\gram} kaszy bulgur
    \item \qty{90}{\milli\litre} tofu
\end{Ingred}

Różyczki brokułu ugotuj wraz z~groszkiem do momentu uzyskania miękkości. Kaszę
gotujemy w~wodzie przez \qty{15}{\minute} do momentu uzyskania miękkości.
Odcedzamy i~odstawiamy do przestygnięcia. Tofu odsączamy z~wody i~kroimy na
paski o~grubości około \qty{1.5}{\centi\metre}. Na patelni rozgrzewamy \num{2}
łyżki oliwy. Układamy tofu i~podsmażamy po \qty{3}{\minute} z~każdej strony.
Przekładamy na talerz. Paprykę pozbawiamy gniazda nasiennego, kroimy w~słupki,
następnie w~niewielką kostkę. Marchew obieramy. Na patelnię wlewamy wodę.
Dodajemy marchewkę i~paprykę. Dusimy pod przykryciem na małym ogniu do momentu
odparowania wody. Na patelnię dodajemy brokuł, groszek oraz ugotowaną kaszę.
Dodajemy \num{1} łyżeczkę oliwy i~wszystkie przyprawy. Całość dokładnie
mieszamy i~podsmażamy na patelni przez około \qtyrange{5}{7}{\minute},
mieszając od czasu do czasu. Kaszę z~warzywami przekładamy do misek. Na
wierzchu układamy kawałki tofu. Podajemy ciepłe.

\recipe{Czekoladowa owsianka -- śniadanie}

\begin{Ingred}
    \item \qty{10}{\gram} orzechów włoskich
    \item \qty{60}{\gram} banana
    \item \qty{10}{\gram} kakao \qty{16}{\percent}
    \item \qty{30}{\gram} płatków owsianych
    \item \qty{110}{\gram} śliwek
    \item \qty{250}{\milli\litre} mleka \qty{1.5}{\percent}
\end{Ingred}

Do mleka wsyp płatki i~kakao, dodaj posiekanego w~plastry banana i~gotuj do
osiągnięcia oczekiwanej gęstości. Owsiankę przerzuć na miseczkę i~dodaj
pozostałe owoce.

\recipe{McWrap z~chrupiącym kurczakiem -- kolacja}

\begin{Ingred}
    \item \qty{100}{\gram} mięsa z~piersi kurczaka
    \item \qty{10}{\gram} mąki pszennej białej
    \item \qty{10}{\milli\litre} oliwy z~oliwek
    \item \qty{10}{\gram} płatków kukurydzianych
    \item \qty{60}{\gram} pomidorów
    \item \qty{1}{\gram} curry w~proszku
    \item \qty{1}{\gram} papryki w~proszku
    \item \qty{1}{\gram} pieprzu czarnego
    \item \qty{20}{\gram} rukoli
    \item \qty{10}{\gram} sałaty lodowej
    \item \qty{1}{\gram} soli
    \item \qty{60}{\gram} tortilli pełnoziarnistej
    \item \qty{15}{\gram} ketchupu
\end{Ingred}

Mięso mieszamy z~oliwą a~następnie obtaczamy w~panierce i~pieczemy około
\qty{20}{\minute} w~\qty{220}{\celsius}. Wrapa smarujemy ketchupem
i~jogurtem/majonezem, nakładamy sałatę, plastry pomidora, upieczone stripsy
i~pokrojony w~paski sera żółtego. Zawijamy.

\paru{Panierka} \qty{10}{\gram} płatków kukurydzianych, \qty{5}{\gram} mąki
kukurydzianych, sól, curry, chili całość blendujemy.

\recipe{Pasta z~wędzonego łososia -- \rom{2} śniadanie}

\begin{Ingred}
    \item \qty{70}{\gram} chleba żytniego
    \item \qty{8}{\gram} kopru ogrodowego
    \item \qty{1}{\gram} pieprzu czarnego
    \item \qty{80}{\gram} sera ricotta
    \item \qty{1}{\gram} soli
    \item \qty{50}{\gram} wędzonego łososia
\end{Ingred}

Do blendera wrzucamy łososia i~ser. Miksujemy do uzyskania gładkiej
konsystencji. Dodajemy sól, pieprz i~koperek po czym mieszamy łyżką. Można
pominąć krok z~blendowaniem i~wymieszać w~miseczce. Dodać dowolne warzywa.

\recipe{Kurczak po toskańsku -- obiad}

\begin{Ingred}
    \item \qty{16}{\gram} parmezanu
    \item \qty{100}{\gram} mięsa z~piersi kurczaka
    \item \qty{5}{\gram} czosnku
    \item \qty{14}{\gram} suszonych pomidorów
    \item \qty{1}{\gram} suszonego oregano
    \item \qty{1}{\gram} pieprzu czarnego
    \item \qty{1}{\gram} soli
    \item \qty{50}{\gram} szpinaku
    \item \qty{50}{\milli\litre} śmietany \qty{18}{\percent}
    \item \qty{100}{\milli\litre} bulionu rosołowego
    \item \qty{50}{\gram} makaronu z~pszenicy durum
\end{Ingred}

Fileta pokroić na kawałki. Całe mięso doprawić solą, pieprzem, oregano, natrzeć
startym czosnkiem i~wysmarować łyżką oliwy. Na tym etapie kurczaku można
odstawić do zamarynowania. Kurczaka dać na rozgrzaną patelnię. Zdjąć kurczaka
z~patelni i~odłożyć na talerz. Na tę samą patelnią wlać bulion i~zagotować.
Włożyć mięso i~obłożyć pokrojonymi suszonymi pomidorami wyjętymi z~zalewy.
Gotować przez około \qty{1}{\minute}, przewrócić na drugą stronę i~powtórzyć
gotowanie. Wlać śmietankę i~zagotować. Posypać szpinakiem i~wcisnąć go w~sos
aby listki zwiędły. Po około \qtyrange{1}{2}{\minute} gotowania, gdy sos będzie
już gęsty, odstawić danie z~patelni. Podawać z~wybranym dodatkiem posypując
tartym serem.

\recipe{Buratta z~roszponką -- kolacja}

\begin{Ingred}
    \item \qty{10}{\gram} miodu pszczelego
    \item \qty{10}{\gram} musztardy
    \item \qty{10}{\milli\litre} oliwy z~oliwek
    \item \qty{10}{\gram} orzechów nerkowca
    \item \qty{100}{\gram} pomidorów koktajlowych
    \item \qty{1}{\gram} pieprzu czarnego
    \item \qty{1}{\gram} soli
    \item \qty{50}{\gram} szpinaku
    \item \qty{5}{\milli\litre} soku z~cytryny
    \item \qty{21}{\gram} marynowanych oliwek zielonych
    \item \qty{30}{\gram} szynki parmeńskiej
    \item \qty{65}{\gram} buratty
\end{Ingred}

Szpinak umyć o~osuszyć. Szynkę porwać na mniejsze kawałki. Oliwki odsączyć
i~pokroić w~paseczki. Z oliwy z~oliwek, miodu i~musztardy przygotować sos.
Wszystkie składniki sałatki wymieszać. Dodać ser. Podprażyć orzechy nerkowca na
suchej patelni.

\recipe{kanapka z~żółtym serem -- śniadanie}

\begin{Ingred}
    \item \qty{70}{\gram} chleba żytniego
    \item \qty{20}{\gram} kiełków rzodkiewki
    \item \qty{80}{\gram} ogórka
    \item \qty{60}{\gram} pomidora
    \item \qty{10}{\gram} masła
    \item \qty{30}{\gram} sera gouda
\end{Ingred}

Na posmarowanym masłem pieczywie ułożyć ser. Pomidora pokroić w~plastry
i~położyć na kanapce.

\recipe{Serek wiejski z~gruszką i~orzechami -- \rom{2} śniadanie}

\begin{Ingred}
    \item \qty{10}{\gram} orzechów włoskich
    \item \qty{130}{\gram} gruszki
    \item \qty{20}{\gram} płatki owsianych
    \item \qty{180}{\gram} serka wiejskiego
\end{Ingred}

Serek wiejski przełożyć do miseczki. Gruszkę pokroić w~kostkę i~ułożyć na
serku, posypać płatkami owsianymi. Posypać orzechami włoskimi. Przed zjedzeniem
wymieszać.

\recipe{Greckie pulpeciki -- obiad}

\begin{Ingred}
    \item \qty{250}{\gram} mrożonych brokułów
    \item \qty{10}{\gram} czosnku
    \item \qty{100}{\gram} ogórka
    \item \qty{5}{\gram} papryki chili
    \item \qty{12}{\gram} natki pietruszki
    \item \qty{10}{\gram} płatków owsianych
    \item \qty{1}{\gram} pieprzu czarnego
    \item \qty{1}{\gram} soli
    \item \qty{100}{\gram} łopatki wieprzowiny
    \item \qty{270}{\gram} ziemniaków
    \item \qty{10}{\milli\litre} wody
    \item \qty{80}{\milli\litre} jogurtu naturalnego Skyr
\end{Ingred}

Mięso mielone umieszczamy w~dużej misce. Dodajemy pokrojoną w~drobną kostkę
cebulę, przeciśnięty przez praskę czosnek, płatki owsiane, posiekaną
pietruszkę, chili oraz wodę. Doprawiamy solą i~pieprzem. Całość mieszamy do
połączenia wszystkich składników. Z masy formujemy kuleczki. Pieczemy
w~piekarniku przez \qty{20}{\minute} w~temperaturze \qty{180}{\celsius}.

Ogórek ścieramy na tarce o~grubych oczkach bezpośrednio na sito nałożoną na
miskę, oprószamy solą, aby wydobyła się woda (najlepiej zostawić tak
przygotowany ogórek na noc w~lodówce). W misce łączymy jogurt, posiekany drobno
koperek, czosnek przeciśnięty przez praskę. Doprawiamy do smaku.

Podawać z~ziemniakami i~brokułem gotowanym (lub dowolną ulubioną
surówką/warzywem.)

\recipe{Włoskie kanapki -- kolacja}

\begin{Ingred}
    \item \qty{7}{\gram} świeżej bazylii
    \item \qty{20}{\gram} cukinii
    \item \qty{10}{\milli\litre} oliwy z~oliwek
    \item \qty{1}{\gram} pieprzu czarnego
    \item \qty{40}{\gram} rukoli
    \item \qty{24}{\gram} salami
    \item \qty{40}{\gram} sera mozzarella
    \item \qty{1}{\gram} soli
    \item \qty{30}{\gram} szynki parmeńskiej
    \item \qty{10}{\gram} czarnych oliwek
\end{Ingred}

Oliwę z~oliwek wymieszać z~czosnkiem i~bazylią. Posmarować bagietkę. Dodać
warzywa i~zapiekać w~piekarniku do rozpuszczenia się sera.

\recipe{Owsianka z~budyniem słony karmel -- śniadanie}

\begin{Ingred}
    \item \qty{60}{\gram} banana
    \item \qty{8}{\gram} migdałów
    \item \qty{30}{\gram} płatków owsianych
    \item \qty{250}{\milli\litre} mleka \qty{1.5}{\percent}
    \item \qty{16}{\gram} budyniu w~proszku
\end{Ingred}

Płatki owsiane zagotować na mleku. Dodać \num{2} łyżki budyniu słony karmel.
Mieszać aż do uzyskania gęstej konsystencji. Dodać banana i~płatki migdałów.

\recipe{Roladki włoskie z~tortilli -- \rom{2} śniadanie}

\begin{Ingred}
    \item \qty{1}{\gram} suszonego oregano
    \item \qty{1}{\gram} pieprzu czarnego
    \item \qty{20}{\gram} rukoli
    \item \qty{20}{\gram} serka naturalnego bieluch
    \item \qty{1}{\gram} soli
    \item \qty{10}{\gram} kiełków lucerny
    \item \qty{15}{\gram} zielonych oliwek marynowanych
    \item \qty{60}{\gram} tortilli pełnoziarnistej
    \item \qty{30}{\gram} szynki parmeńskiej
\end{Ingred}

Tortillę posmarować serkiem, dodać szynkę, następnie dodać liście rukoli,
plasterki pomidora i~kiełki. Posypać suszonym oregano. Doprawić solą
i~pieprzem, zwinąć tortille w~miarę ciasno w~rulonik. Pokroić na porcje.

\recipe{Mintaj na warzywach}

\begin{Ingred}
    \item \qty{5}{\gram} czosnku
    \item \qty{150}{\gram} mintaja
    \item \qty{120}{\gram} ogórka kwaszonego
    \item \qty{10}{\milli\litre} oliwy z~oliwek
    \item \qty{6}{\gram} natki pietruszki
    \item \qty{1}{\gram} papryki chili w~proszku
    \item \qty{1}{\gram} pieprzu czarnego
    \item \qty{50}{\gram} ryżu białego
    \item \qty{1}{\gram} soli
    \item \qty{100}{\gram} warzyw na patelni Hortex
    \item \qty{6}{\milli\litre} soku z~cytryny
    \item \qty{30}{\gram} sera gouda
\end{Ingred}

Rybę przyprawić solą, papryką i~pieprzem oraz skropić sokiem z~cytryny. Na
rozgrzaną oliwę wrzucić mieszankę warzyw. Dusić około \qty{10}{\minute}
następnie dodać posiekany czosnek i~ułożyć rybę, przykryć, zmniejszyć gaz
i~dusić jeszcze około \qty{10}{\minute}. Na sam koniec duszenia ułożyć na rybie
plasterki sera i~odstawić pod przykryciem, aż ser się nie rozpuści. Posypać
natką pietruszki. Podawać z~ryżem i~ogórkami.

\recipe{Zupa krem z~soczewicy z~mlekiem kokosowym -- podwieczorek}

\begin{Ingred}
    \item \qty{23}{\gram} czerwonej soczewicy, suche nasiona
    \item \qty{125}{\gram} bulionu warzywnego
    \item \qty{3}{\gram} cebuli
    \item \qty{13}{\gram} marchwi
    \item \qty{5}{\milli\litre} oliwy z~oliwek
    \item \qty{3}{\gram} natki pietruszki
    \item \qty{23}{\gram} pora
    \item \qty{1}{\gram} suszonego kminu rzymskiego
    \item \qty{1}{\gram} kolendry
    \item \qty{1}{\gram} kurkumy
    \item \qty{1}{\gram} liścia laurowego
    \item \qty{1}{\gram} papryki w~proszku
    \item \qty{1}{\gram} pieprzu czarnego
    \item \qty{100}{\gram} wczesnych ziemniaków
    \item \qty{2}{\gram} ziela angielskiego
    \item \qty{25}{\milli\litre} mleczka kokosowego
\end{Ingred}

W garnku na oliwie zeszklić pokrojoną w~kosteczkę cebulę, po chwili dodać
pokrojone w~kosteczkę ząbki czosnku, a~następnie pokrojonego na małe kawałki
pora (białą część). Dodać kmin oraz nasiona kolendry. Smażyć na umiarkowanym
ogniu przez około \qty{2}{\minute}, co chwilę mieszając. Dodać obraną i~startą
na tarce marchewkę, obrane i~pokrojone w~małą kosteczkę ziemniaki oraz wsypać
suchą soczewicę. Doprawić solą, świeżo zmielonym pieprzem, dodać kurkumę,
paprykę w~proszku i~ziele angielskie. Wymieszać, zalać gorącym bulionem,
przykryć i~gotować przez około \qty{15}{\minute}, aż składniki będą bardzo
miękkie. Na koniec dodać mleko i~podgrzać. Przed podaniem zblendować. Podawać
z~kolendrą, pietruszką i~oliwią.

\recipe{Grzanki z~ricottą -- kolacja}

\begin{Ingred}
    \item \qty{70}{\gram} chleba żytniego
    \item \qty{200}{\gram} pomidorów koktajlowych
    \item \qty{1}{\gram} pieprzu czarnego
    \item \qty{20}{\gram} rukoli
    \item \qty{80}{\gram} sera ricotta
\end{Ingred}

Chleb grillujemy na patelni/grillu elektrycznym. Smarujemy serem ricotta.
Dodajemy pomidorki koktajlowe oraz rukolę. Doprawiamy do smaku.

\recipe{Tost z~szynką, mozzarellą i~papryką -- śniadanie}

\begin{Ingred}
    \item \qty{70}{\gram} chleba żytniego
    \item \qty{100}{\gram} papryki czerwonej
    \item \qty{30}{\gram} sera mozzarella
    \item \qty{30}{\gram} szynki z~piersi kurczaka
\end{Ingred}

Tost zapiecz w~tosterze. Do środka dodaj wędlinę, ser i~paprykę.

\recipe{Koktajl malinowo--biszkoptowy -- \rom{2} śniadanie}

\begin{Ingred}
    \item \qty{15}{\gram} biszkopty bez cukrowe Mamut
    \item \qty{10}{\gram} czekolady gorzkiej (kostka)
    \item \qty{200}{\gram} mrożonych malin
    \item \qty{10}{\gram} płatków owsianych
    \item \qty{250}{\milli\litre} maślanki nisko tłuszczowej
\end{Ingred}

Zmiksować wszystko na gładką masę. Czekoladę można pokroić na kawałki i~posypać
na wierzch.

\recipe{Zupa krem marchewkowo--imbirowa -- podwieczorek}

\begin{Ingred}
    \item \qty{50}{\gram} cebuli
    \item \qty{5}{\gram} czosnku
    \item \qty{3}{\gram} korzenia imbiru
    \item \qty{120}{\gram} marchwi
    \item \qty{15}{\gram} migdałów
    \item \qty{10}{\milli\litre} marchwi
    \item \qty{1}{\gram} pieprzu czarnego
    \item \qty{5}{\gram} skórki limonki
    \item \qty{1}{\gram} soli
\end{Ingred}

Na rozgrzanej oliwie zeszklić poszatkowaną cebulę i~czosnek. Dodać pokrojoną
w~plasterki marchewkę, starty imbir i~skórkę limonki. Całość zalać wodą (około
\num{2} szklanki) i~dusić przez \qty{1}{\hour}. Po ugotowaniu zupę zmiksować
oraz doprawić solą i~pieprzem. Podawać z~uprażonymi płatkami migdałowymi.

\recipe{Chińska wołowina z~groszkiem -- obiad}

\begin{Ingred}
    \item \qty{100}{\gram} groszku cukrowego
    \item \qty{1}{\gram} korzenia imbiru
    \item \qty{8}{\gram} mąki kukurydzianej
    \item \qty{10}{\milli\litre} oleju rzepakowego
    \item \qty{3}{\gram}  papryki chili
    \item \qty{10}{\milli\litre} sosu sojowego
    \item \qty{100}{\milli\litre} bulionu rosołowego
    \item \qty{100}{\gram} łopatki wołowej
    \item \qty{50}{\gram} ryżu jaśminowego
\end{Ingred}

Wołowinę umyć w~zimnej wodzie, osuszyć. Bardzo ostrym nożem kroić mięso
w~cienkie plasterki i~włożyć do miski. Mieszać razem sos sojowy i~starty imbir,
wlać do mięsa i~starannie mieszać. Przykryć i~odstawić na bok. Groszek cukrowy
starannie opłukać na sicie i~równie starannie osuszyć, przyciąć z~obu stron.
Papryczkę chili pokroić wzdłuż na pół, usunąć pestki i~ogonek, umyć i~osuszyć.
Pokroić papryczkę w~długie, cienkie paski. Mąkę kukurydzianą rozprowadzić
zimnym bulionem, mieszać, żeby nie było grudek. Postawić wok na naprawdę dużym
ogniu, a~kiedy zacznie dymić, wlać olej. Wrzucić mięso i~groszek cukrowy
i~smażyć na dużym ogniu, mieszając przez \qtyrange{2}{3}{\minute}, aż mięso
zmieni kolor, a~nawet zacznie się lekko rumienić. Wlać wywar z~mąką. Mieszać,
po czym gotować na dużym ogniu, cały czas mieszając, aż sos zgęstnieje
i~pokryje kawałki mięsa. Zdjąć wok z~ognia, przełożyć mięso do miski i~posypać
paskami chilli. Można użyć mrożonki warzyw chińskich.

\recipe{Pizza na placku tortilli -- kolacja}

\begin{Ingred}
    \item \qty{30}{\gram} koncentratu pomidorowego \qty{30}{\percent}
    \item \qty{100}{\gram} pomidorów koktajlowych
    \item \qty{1}{\gram} suszonej bazylii
    \item \qty{1}{\gram} suszonego oregano
    \item \qty{20}{\gram} rukoli
    \item \qty{60}{\gram} tortilli pełnoziarnistej
    \item \qty{20}{\gram} czarnych oliwek
    \item \qty{60}{\gram} buratty
\end{Ingred}

Posmarować tortillę sosem, położyć ser i~dodatki. Piec na suchej patelni do
momentu roztopienia sera. Można posypać ziołami do smaku. Dodać rukolę
i~pomidorki koktajlowe.

\recipe{Warzywne kanapki -- śniadanie}

\begin{Ingred}
    \item \qty{10}{\gram} szczypiorku
    \item \qty{70}{\gram} chleba żytniego
    \item \qty{70}{\gram} pomidorów
    \item \qty{1}{\gram} pieprzu czarnego
    \item \qty{30}{\gram} rzodkiewki
    \item \qty{1}{\gram} soli
    \item \qty{10}{\gram} masła
    \item \qty{10}{\gram} szynki z~piersi kurczaka
    \item \qty{10}{\gram} sałaty zielonej
\end{Ingred}

Posmaruj kanapki masłem, dodaj wędlinę i~spożywaj z~warzywami.

\recipe{Bananowe--kakaowe pancakes -- \rom{2} śniadanie}

\begin{Ingred}
    \item \qty{50}{\gram} jajek
    \item \qty{120}{\gram} banana
    \item \qty{5}{\gram} kakao \qty{16}{\percent}
    \item \qty{100}{\gram} mrożonych malin
    \item \qty{160}{\milli\litre} jogurtu typu islandzkiego waniliowego Skyr
    \item \qty{20}{\gram} masła orzechowego z~kawałkami orzechów
\end{Ingred}

Banana rozgnieść widelcem, dodać jajko, kakao i~roztrzepać na jednolitą masę.
Na patelni usmażyć niewielkie placuszki (pancakes). Gotowe pancakes posmarować
masłem orzechowym i~posypać malinami.

\recipe{Spaghetti z~cukinią i~ricottą -- obiad}

\begin{Ingred}
    \item \qty{3}{\gram} świeżej bazylii
    \item \qty{150}{\gram} cukinii
    \item \qty{5}{\gram} czosnku
    \item \qty{50}{\gram} makaronu pełnoziarnistego
    \item \qty{10}{\milli\litre} oliwy z~oliwek
    \item \qty{1}{\gram} pieprzu czarnego
    \item \qty{100}{\gram} sera ricotta
    \item \qty{1}{\gram} soli
    \item \qty{1}{\gram} gałki muszkatołowej
\end{Ingred}

Cukinie ścieramy na tarce o~dużych oczkach, solimy i~odstawiamy na
\qty{10}{\minute}. Posikany czosnek podsmażamy na oliwie, dodajemy cukinię
i~smażymy około \qty{5}{\minute}. Doprawiamy gałką muszkatołową i~pieprzem.
Cukinie zdejmujemy z~ognia, dodajemy pokruszoną ricottę i~świeżą bazylię.
Makaron gotujemy Al Dente i~mieszamy z~cukinią.

\recipe{Sałatka z~kurczakiem w~migdałach -- kolacja}

\begin{Ingred}
    \item \qty{8}{\gram} sera parmezan
    \item \qty{80}{\gram} mięsa z~piersi kurczaka
    \item \qty{20}{\gram} czerwonej cebuli
    \item \qty{15}{\gram} migdałów
    \item \qty{10}{\milli\gram} oliwy z~oliwek
    \item \qty{120}{\gram} pomarańczy
    \item \qty{100}{\gram} pomidorów koktajlowych
    \item \qty{1}{\gram} papryki chili w~proszku
    \item \qty{1}{\gram} curry
    \item \qty{1}{\gram} pieprzu czarnego
    \item \qty{40}{\gram} rukoli
    \item \qty{1}{\gram} soli
\end{Ingred}

Kurczaka kroimy w~wąskie i~długie paski i~marynujemy w~oleju, soli, pieprzu
curry i~ostrej papryce, odstawiamy do lodówki na co najmniej \qty{1}{\hour}.
Kawałki mięsa obtaczamy w~pokruszonych płatkach migdałów, układamy w~naczyniu
żaroodpornym i~pieczemy w~temperaturze \qty{180}{\celsius} przez około
\qtyrange{15}{20}{\minute}. Pomarańczę obieramy, dzielimy na cząstki, kroimy na
drobne kawałki i~usuwamy białe włókna. Na talerzu rozkładamy umytą sałatę, owoc
oraz upieczonego kurczaka. Dodajemy pomidorki koktajlowe, cebulę i~posypujemy
całość parmezanem.

\recipe{Jajka na miękko z~kanapkami -- śniadanie}

\begin{Ingred}
    \item \qty{100}{\gram} jajek
    \item \qty{10}{\gram} szczypiorku
    \item \qty{70}{\gram} chleba żytniego
    \item \qty{70}{\gram} ogórka
    \item \qty{130}{\gram} pomidorów
    \item \qty{30}{\gram} szynki z~indyka
    \item \qty{10}{\gram} masła
\end{Ingred}

Jajko włożyć do wrzątku, gotować przez \qty{5}{\minute}. Pieczywo posmarować
masłem, dodać wędlinę i~ułożyć na nim pokrojonego w~plasterki pomidora
i~ogórka, posypać szczypiorkiem.

\recipe{Jogurt z~granolą domową -- \rom{2} śniadanie}

\begin{Ingred}
    \item \qty{6}{\gram} wiórków kokosowych
    \item \qty{5}{\milli\litre} oleju kokosowego
    \item \qty{30}{\gram} płatków owsianych
    \item \qty{70}{\gram} winogron
    \item \qty{8}{\gram} suszonych jagód goji
    \item \qty{180}{\gram} jogurtu naturalnego \qty{2}{\percent}
\end{Ingred}

Na patelni rozpuścić olej kokosowy. Wsypać płatki. Smażyć na niedużym ogniu,
mieszając, aż się płatki lekko zarumienią. Zdjąć z~ognia, wsypać jagody goji.
Wymieszać i~przełożyć na głęboki talerz do ostudzenia. Wymieszać granolę
z~jogurtem naturalnym. Posypać płatkami migdałów. Jagody goji można zastąpić
rodzynkami lub daktylami.

\recipe{Kurczak po koreańsku -- obiad}

\begin{Ingred}
    \item \qty{100}{\gram} mięsa z~piersi kurczaka
    \item \qty{10}{\gram} sezamu
    \item \qty{100}{\gram} brokuła
    \item \qty{5}{\gram} cukru
    \item \qty{50}{\gram} cukinii
    \item \qty{50}{\gram} marchwi
    \item \qty{10}{\milli\litre} oleju rzepakowego
    \item \qty{1}{\gram} papryki chili w~proszku
    \item \qty{1}{\gram} czosnku granulowanego
    \item \qty{50}{\gram} ryżu
    \item \qty{10}{\milli\litre} sosu sojowego
\end{Ingred}

Piekarnik nagrzać do \qty{200}{\celsius}. Przygotować blaszkę do pieczenia
z~nieprzywierającą powłoką, lub wyłożyć ją papierem do pieczenia. Piersi
z~kurczaka pokroić wzdłuż na mniejsze kawałki. Marchewkę i~cukinię pokroić pod
kątem na cienkie plasterki. Brokuła pokroić na małe różyczki. Pokrojone warzywa
przełożyć do miski, do osobnej miski przełożyć kurczaka.

W małej misce wymieszać do uzyskania gładkiej konsystencji: sos sojowy, brązowy
cukier, olej chili i~suszony czosnek. Wymieszać warzywa oraz kurczaka w~sosie.

Kurczaka wyłożyć na środek blaszki i~dookoła wyłożyć warzywa, posypać po
wierzchu sezamem. Piec \qtyrange{25}{30}{\minute} aż kurczak się zarumieni
a~warzywa będą miękkie, sprawdzić widelcem. W połowie czasu pieczenia, kurczaka
i~warzywa obrócić na drugą stronę. Podawać z~ryżem

\recipe{Sałatka z~serka wiejskiego, tuńczyka, papryki i~ogórka -- kolacja}

\begin{Ingred}
    \item \qty{35}{\gram} chleba żytniego
    \item \qty{20}{\gram} kiełków rzodkiewki
    \item \qty{60}{\gram} ogórka
    \item \qty{100}{\gram} czerwonej papryki
    \item \qty{1}{\gram} pieprzu czarnego
    \item \qty{30}{\gram} rzodkiewki
    \item \qty{1}{\gram} soli
    \item \qty{60}{\gram} tuńczyka w~sosie własnym
    \item \qty{180}{\gram} serka wiejskiego
\end{Ingred}

Do serka dodać tuńczyka, warzywa, doprawić do smaku i~wymieszać.

\recipe{Croissant wytrawny -- śniadanie}

\begin{Ingred}
    \item \qty{1}{\gram} świeżej bazylii
    \item \qty{55}{\gram} czerwonej papryki
    \item \qty{10}{\gram} szpinaku
    \item \qty{30}{\gram} szynki z~indyka
    \item \qty{15}{\gram} sera gouda
    \item \qty{60}{\gram} croissanta
\end{Ingred}

Rozgrzać patelnie. Na przekrojonego croissanta dodać ser, szynkę i~dodatki
warzywne. Zapiekać do rozpuszczenia się sera.

\recipe{Koktajl truskawkowy -- \rom{2} śniadanie}

\begin{Ingred}
    \item \qty{20}{\gram} płatków owsianych
    \item \qty{300}{\gram} mrożonych truskawek
    \item \qty{180}{\gram} jogurtu naturalnego \qty{2}{\percent}
    \item \qty{7}{\gram} ksylitolu
\end{Ingred}

Zmiksować wszystko na gładką masę.

\recipe{Spaghetti z~sosem z~czerwonej soczewicy -- obiad}

\begin{Ingred}
    \item \qty{40}{\gram} czerwonej soczewicy
    \item \qty{25}{\gram} cebuli
    \item \qty{3}{\gram} czosnku
    \item \qty{50}{\gram} makaronu pełnoziarnistego
    \item \qty{10}{\milli\litre} oliwy z~oliwek
    \item \qty{170}{\gram} pomidorów
    \item \qty{35}{\gram} suszonych pomidorów
    \item \qty{1}{\gram} pieprzu
    \item \qty{1}{\gram} soli
\end{Ingred}

Do gotującej się wody wsypać opłukaną soczewicę i~gotować ją przez
\qty{15}{\minute}. Cebulę i~czosnek pokroić i~zeszklić na patelni z~oliwą,
następnie dodać do ugotowanej soczewicy. Do garnka wrzucić także pokrojone
pomidory (świeże i~suszone) i~przyprawy, następnie zblendować na gładki sos
i~tak przygotowany gotować około \qty{15}{\minute}. Gotowym sosem polać
ugotowany makaron.

\recipe{Chia pudding z~malinami -- podwieczorek}

\begin{Ingred}
    \item \qty{100}{\gram} mrożonych malin
    \item \qty{180}{\gram} jogurtu naturalnego \qty{2}{\percent}
    \item \qty{20}{\gram} nasion chia
    \item \qty{7}{\gram} ksylitolu
\end{Ingred}

Nasiona chia wymieszać z~jogurtem. Maliny rozmrozić. Zblendować je na sos.
Polać sosem pudding chia i~wymieszać z~ksylitolem.

\recipe{Meksykańska sałatka -- kolacja}

\begin{Ingred}
    \item \qty{30}{\gram} cebuli
    \item \qty{35}{\gram} chleba żytniego
    \item \qty{5}{\gram} czosnku
    \item \qty{40}{\gram} jogurtu naturalnego \qty{2}{\percent}
    \item \qty{1}{\gram} suszonej papryki chili
    \item \qty{240}{\gram} czerwonej papryki
    \item \qty{50}{\gram} pora
    \item \qty{1}{\gram} pieprzu czarnego
    \item \qty{1}{\gram} soli
    \item \qty{15}{\gram} sera gouda
    \item \qty{40}{\gram} konserwowej kukurydzy
    \item \qty{60}{\gram} konserwowej czerwonej fasoli
    \item \qty{80}{\gram} wędzonej piersi z~kurczaka
\end{Ingred}

Paprykę pokroić, wymieszać z~kukurydzą, czerwoną fasolą, żółtym serem, wędzonym
kurczakiem. Pokroić por w~talarki. Czosnek przecisnąć przez praską i~wymieszać
z~jogurtem, pieprzem, solą i~chili. Połączyć wszystkie składniki, polać sosem
jogurtowo--czosnkowym. Zjeść z~kromką pieczywa.

\recipe{Granola z~brzoskwinią i~migdałami -- śniadanie}

\begin{Ingred}
    \item \qty{80}{\gram} brzoskwiń
    \item \qty{15}{\gram} migdałów
    \item \qty{30}{\gram} płatków owsianych
    \item \qty{1}{\gram} mielonego cynamonu
    \item \qty{10}{\gram} miodu
    \item \qty{160}{\milli\litre} jogurtu naturalnego Skyr
\end{Ingred}

Migdały podprażamy na suchej patelni, studzimy, kroimy nożem na małe kawałki.
Dorzucamy płatki owsiane, odrobinę miodu i~opcjonalnie cynamon. Podajemy
z~jogurtem naturalnym i~brzoskwinią.

\recipe{Tortilla z~hummusem -- \rom{2} śniadanie}

\begin{Ingred}
    \item \qty{20}{\gram} kiełków rzodkiewki
    \item \qty{60}{\gram} ogórka
    \item \qty{100}{\gram} czerwonej papryki
    \item \qty{1}{\gram} pieprzu czarnego
    \item \qty{20}{\gram} rukoli
    \item \qty{1}{\gram} soli
    \item \qty{30}{\gram} szynki z~piersi kurczaka
    \item \qty{50}{\gram} hummusu
    \item \qty{60}{\gram} tortilli pełnoziarnistej
\end{Ingred}

Tortille posmaruj hummusem, dodaj ogórka i~porwane liście sałaty oraz resztę
dodatków, zawiń. Grilluj na rozgrzanej patelni z~dwóch stron lub pokrój na małe
rulony.

\recipe{Makaron z~pesto -- obiad}

\begin{Ingred}
    \item \qty{200}{\gram} brokuła
    \item \qty{60}{\gram} makaronu pełnoziarnistego
    \item \qty{15}{\gram} migdałów
    \item \qty{100}{\gram} pomidorów koktajlowych
    \item \qty{2}{\gram} pieprzu czarnego
    \item \qty{1}{\gram} soli
    \item \qty{20}{\gram} sosu pesto
    \item \qty{100}{\gram} wędzonej piersi z~kurczaka
\end{Ingred}

Makaron ugotować Al Dente. Wymieszać z~pesto, ugotować brokułem i~wędzonym
kurczakiem. Doprawić do smaku. Dodać pomidorki koktajlowe. Dosypać prażonymi
płatkami migdałów.

\recipe{Pasta z~makreli -- kolacja}

\begin{Ingred}
    \item \qty{50}{\gram} czerwonej cebuli
    \item \qty{70}{\gram} chleba żytniego
    \item \qty{20}{\gram} jogurtu naturalnego \qty{2}{\percent}
    \item \qty{70}{\gram} wędzonej makreli
    \item \qty{1}{\gram} pieprzu czarnego
    \item \qty{1}{\gram} soli
    \item \qty{120}{\gram} ogórków kiszonych
\end{Ingred}

Makrelę dokładnie obieramy ze wszystkich ości, ogórka i~cebulkę kroimy w~drobną
kostkę, wszystko ze sobą łączymy i~dodajemy łyżkę jogurtu naturalnego
i~doprawiamy solą i~pieprzem. Odstawić na \qty{1.5}{\hour} do lodówki.

\recipe{Kanapka z~serem żółtym, wędliną i~roszponką -- śniadanie}

\begin{Ingred}
    \item \qty{70}{\gram} chleba żytniego
    \item \qty{60}{\gram} rzodkiewki
    \item \qty{10}{\gram} masła
    \item \qty{30}{\gram} szynka z~piersi kurczaka
    \item \qty{30}{\gram} sera gouda
    \item \qty{20}{\gram} roszponki
\end{Ingred}

Chleb posmarować delikatnie masłem, położyć plaster sera, wędlinę, roszponkę
i~rzodkiewkę.

\recipe{letnia sałatka owocowa -- \rom{2} śniadanie}

\begin{Ingred}
    \item \qty{350}{\gram} arbuza
    \item \qty{50}{\gram} borówki amerykańskiej
    \item \qty{10}{\gram} gorzkiej czekolady
    \item \qty{120}{\gram} nektarynki
    \item \qty{10}{\gram} płatki owsianych
    \item \qty{50}{\gram} truskawek
    \item \qty{250}{\milli\litre} kefiru
\end{Ingred}

Arbuza pokroić na kawałki, dodać nektarynkę, truskawki oraz borówkę
amerykańską. Dodać drobno posiekaną gorzką czekoladę, zalać kefirem i~posypać
łyżką płatków owsianych.

\recipe{Pierś kurczaka faszerowaną -- obiad}

\begin{Ingred}
    \item \qty{100}{\gram} mięsa z~piersi kurczaka
    \item \qty{5}{\gram} oliwy z~oliwek
    \item \qty{21}{\gram} pomidorów suszonych
    \item \qty{1}{\gram} suszonej bazylii
    \item \qty{1}{\gram} pieprzu czarnego
    \item \qty{40}{\gram} sera mozzarella
    \item \qty{1}{\gram} soli
    \item \qty{270}{\gram} wczesnych ziemniaków
    \item \qty{200}{\gram} fasolki szparagowej
\end{Ingred}

W piersi zrób dziurę w~kształcie kieszonki. Mozzarellę i~pomidory suszone
pokrój w~plastry i~włóż do środka kurczaka. Pierś wcześniej posyp ziołami.
Ziemniaki ugotuj. Pierś z~kurczaka przełóż do rękawa do pieczenia i~piecz około
\qtyrange{45}{50}{\minute} w~temperaturze \qty{180}{\celsius}. Całe danie
podawaj z~ziemniakami i~ugotowaną fasolką.

\recipe{Sałatka grecka -- kolacja}

\begin{Ingred}
    \item \qty{25}{\gram} czerwonej cebuli
    \item \qty{35}{\gram} chleba żytniego
    \item \qty{10}{\gram} musztardy
    \item \qty{90}{\gram} ogórków
    \item \qty{100}{\gram} czerwonej papryki
    \item \qty{170}{\gram} pomidorów
    \item \qty{1}{\gram} suszonego majeranku
    \item \qty{1}{\gram} suszonego oregano
    \item \qty{1}{\gram} pieprzu czarnego
    \item \qty{70}{\gram} sera feta
    \item \qty{10}{\gram} pestek słonecznika
    \item \qty{6}{\milli\litre} soku z~cytryny
    \item \qty{10}{\gram} miodu
    \item \qty{50}{\gram} sałaty zielonej
    \item \qty{15}{\gram} oliwek zielonych
\end{Ingred}

Ogórek i~pomidor pokroić w~kostkę, cebulę czerwoną w~piórka, oliwki przekroić
na pół. Warzywa połączyć z~sałatą, przyprawić pieprzem, majerankiem i~oregano.
Dodać kostki sera feta i~delikatnie wymieszać. Sałatkę posypać uprażonymi na
suchej patelni pestkami słonecznika, całość polać sosem z~musztardy, soku
z~cytryny i~miodu.

\recipe{Owsianka budyniowa -- śniadanie}

\begin{Ingred}
    \item \qty{100}{\gram} malin
    \item \qty{8}{\gram} migdałów
    \item \qty{30}{\gram} płatków owsianych
    \item \qty{250}{\milli\litre} mleka \qty{1.5}{\percent}
    \item \qty{16}{\gram} budyniu w~proszku
\end{Ingred}

Płatki owsiane zagotować na mleku. Dodać \num{2} łyżki budyniu bez cukru.
Mieszać aż do uzyskania gęstej konsystencji. Dodać maliny i~płatki migdałów.

\recipe{Sałatka z~serka wiejskiego -- \rom{2} śniadanie}

\begin{Ingred}
    \item \qty{5}{\gram} szczypiorku
    \item \qty{70}{\gram} chleba żytniego
    \item \qty{90}{\gram} ogórka
    \item \qty{1}{\gram} pieprzu czarnego
    \item \qty{45}{\gram} rzodkiewki
    \item \qty{1}{\gram} soli
    \item \qty{180}{\gram} serka wiejskiego
\end{Ingred}

Szczypiorek, ogórki i~rzodkiewki pokroić. Wymieszać z~serkiem. Podawać na
kanapce.

\recipe{Młode ziemniaki z~jajkiem sadzonym -- obiad}

\begin{Ingred}
    \item \qty{100}{\gram} jajek
    \item \qty{60}{\gram} jogurtu naturalnego \qty{2}{\percent}
    \item \qty{200}{\gram} ogórka
    \item \qty{10}{\milli\litre} oliwy z~oliwek
    \item \qty{1}{\gram} pieprzu czarnego
    \item \qty{1}{\gram} soli
    \item \qty{8}{\gram} kopru
    \item \qty{270}{\gram} ziemniaków
\end{Ingred}

Jajko wbić na rozgrzaną patelnię z~odrobiną oliwy. Gotowe jajko sadzone podawać
z~ziemniakami i~mizerią. Ogórka obrać, pokroić w~plasterki, wymieszać
z~jogurtem i~przyprawić do smaku i~koperkiem

\recipe{Pizza na placku tortilli -- kolacja}

\begin{Ingred}
    \item \qty{30}{\gram} koncentratu pomidorowego \qty{30}{\percent}
    \item \qty{100}{\gram} pomidorów koktajlowych
    \item \qty{1}{\gram} suszonej bazylii
    \item \qty{1}{\gram} suszonej oregano
    \item \qty{20}{\gram} rukoli
    \item \qty{40}{\gram} sera mozzarella
    \item \qty{24}{\gram} salami wieprzowego
    \item \qty{60}{\gram} tortilli pełnoziarnistej
\end{Ingred}

Posmarować tortillę sosem, położyć ser, salami i~dodatki. Piec na suchej
patelni do momentu roztopienia sera. Dodać rukolę i~pomidorki koktajlowe.

\recipe{Kanapki z~pastą twarogową -- śniadanie}

\begin{Ingred}
    \item \qty{2}{\gram} szczypiorku
    \item \qty{70}{\gram} chleba żytniego
    \item \qty{20}{\gram} jogurtu naturalnego \qty{2}{\percent}
    \item \qty{180}{\gram} ogórka
    \item \qty{1}{\gram} pieprzu czarnego
    \item \qty{100}{\gram} sera twarogowego półtłustego
    \item \qty{1}{\gram} soli
\end{Ingred}

Ogórka małosolnego pokroić w~kostkę, wymieszać z~serem, szczypiorkiem
i~jogurtem naturalnym. Doprawić do smaku.

\recipe{Koktajl arbuzowo--truskawkowy -- \rom{2} śniadanie}

\begin{Ingred}
    \item \qty{150}{\gram} arbuza
    \item \qty{10}{\gram} pestek słonecznika
    \item \qty{50}{\gram} truskawek
    \item \qty{5}{\gram} mięty
    \item \qty{170}{\gram} jogurtu naturalnego \qty{2}{\percent}
\end{Ingred}

Do jogurtu wsypać słonecznik, dodać truskawki miąższ arbuza pozbawiony pestek,
całość zblendować do uzyskania konsystencji koktajlu. Można zmiksować z~miętą
dla lepszego smaku.

\recipe{Gulasz z~indyka -- obiad}

\begin{Ingred}
    \item \qty{150}{\gram} mięso z~udka indyka
    \item \qty{5}{\gram} czosnku
    \item \qty{50}{\gram} kaszy gryczanej
    \item \qty{4}{\gram} kopru
    \item \qty{120}{\gram} marchwi
    \item \qty{10}{\milli\litre} oliwy z~oliwek
    \item \qty{45}{\gram} czerwonej papryki
    \item \qty{80}{\gram} pieczarek
    \item \qty{1}{\gram} liścia laurowego
    \item \qty{1}{\gram} pieprzu czarnego
    \item \qty{3}{\gram} soli
    \item \qty{100}{\gram} włoszczyzny krojonej w~paski
    \item \qty{1}{\gram} ziela angielskiego mielonego
    \item \qty{180}{\gram} ogórków kiszonego
\end{Ingred}

Indyka umyć, osuszyć, pokroić w~kostkę przyprawić solą, pieprzem, natrzeć
rozgniecionym czosnkiem. Poddusić przez chwilę na patelni bez tłuszczu. W
garnku zagotować wodę z~liściem laurowym i~zielem angielskim. Wrzucić mięso
i~gotować przez około \qty{30}{\minute}, aż będzie miękkie. Warzywa umyć
i~osuszyć. Paprykę pokroić w~kostkę, pieczarki w~plasterki. Mniej więcej
w~połowie czasu gotowania mięsa wrzucić pokrojone warzywa i~rozmrożoną
włoszczyznę. Pod koniec potrawę posypać posiekanym koperkiem. Podawać z~kaszą
i~ogórkami małosolnymi lub kiszonymi.

\recipe{Sałatka caprese -- kolacja}

\begin{Ingred}
    \item \qty{10}{\gram} bazylii
    \item \qty{35}{\gram} chleba żytniego
    \item \qty{10}{\milli\litre} oliwy z~oliwek
    \item \qty{170}{\gram} pomidorów
    \item \qty{2}{\gram} pieprzu czarnego
    \item \qty{60}{\gram} sera mozzarella
    \item \qty{1}{\gram} soli
\end{Ingred}

Mozzarellę oraz pomidora pokroić w~plasterki. Pomidora można obrać ze skóry.
Polać oliwą, dodać pieprz i~odrobinę soli, udekorować listkami bazylii.

\recipe{Parówki na ciepło -- śniadanie}

\begin{Ingred}
    \item \qty{70}{\gram} chleba żytniego
    \item \qty{150}{\gram} pomidorów
    \item \qty{10}{\gram} masła
    \item \qty{20}{\gram} sałaty zielonej
    \item \qty{80}{\gram} parówek
    \item \qty{15}{\gram} ketchupu
\end{Ingred}

Parówki zagotować w~wodzie. Zjeść z~pieczywem oraz warzywami.

\recipe{Koktajl owoce leśne -- \rom{2} śniadanie}

\begin{Ingred}
    \item \qty{100}{\gram} borówki amerykańskiej
    \item \qty{100}{\gram} malin
    \item \qty{250}{\milli\litre} maślanki
    \item \qty{10}{\gram} nasion chia
    \item \qty{7}{\gram} ksylitolu
\end{Ingred}

Wszystko zmiksować.

\recipe{Dorsz w~cytrynie -- obiad}

\begin{Ingred}
    \item \qty{40}{\gram} cytryny
    \item \qty{10}{\gram} czosnku
    \item \qty{120}{\gram} dorsza
    \item \qty{100}{\gram} frytek mrożonych
    \item \qty{4}{\gram} kopru
    \item \qty{10}{\milli\litre} oliwy z~oliwek
    \item \qty{1}{\gram} papryki w~proszku
\end{Ingred}

W małej miseczce wymieszać ze sobą oba rodzaje papryki, koperek, czosnek oraz
sól. Dodać oliwę z~oliwek i~połączyć, aż powstanie pasta. Rybę umyć, osuszyć
papierowym ręcznikiem i~dokładnie natrzeć powstałą pastą. Filety ułożyć na
blasze wyłożonej folią aluminiową lub papierem do pieczenia. Wierzch ryby
skropić odrobiną soku z~cytryny. Piekarnik rozgrzać do \qty{200}{\celsius}.
Piec rybę około \qtyrange{7}{10}{\minute}, aż mięso wewnątrz stanie się białe,
a~wierzch ryby się zarumieni. Frytki piec z~rybą, natomiast wymagają one więcej
czasu niż ryba, sugerować się czasem na opakowaniu.

\recipe{Surówka z~marchewki}

\begin{Ingred}
    \item \qty{5}{\gram} cukru
    \item \qty{90}{\gram} jabłka
    \item \qty{90}{\gram} marchwi
    \item \qty{5}{\milli\litre} oleju rzepakowego
    \item \qty{1}{\gram} soli
    \item \qty{6}{\milli\litre} soku z~cytryny
\end{Ingred}

Marchew i~jabłko obrać, zetrzeć na tarce. Całość wymieszać, opcjonalnie można
dodać sok z~cytryny oraz łyżkę oleju rzepakowego.

\recipe{Burgery drobiowe -- kolacja}

\begin{Ingred}
    \item \qty{15}{\gram} bazylii
    \item \qty{100}{\gram} bułki grahamki
    \item \qty{20}{\gram} cebuli
    \item \qty{10}{\gram} czosnku
    \item \qty{5}{\gram} korzeń imbiru
    \item \qty{60}{\gram} mięsa z~piersi indyka
    \item \qty{50}{\gram} pomidorów
    \item \qty{5}{\gram} liści kolendry
    \item \qty{15}{\gram} sera goudy
    \item \qty{10}{\gram} sałaty zielonej
    \item \qty{30}{\gram} ogórków kiszonych
    \item \qty{15}{\gram} ketchupu
\end{Ingred}

Mięso opłukać, oczyścić i~zmielić. Dodać starty imbir, przeciśnięty przez
praskę czosnek, drobno posiekaną cebulę oraz kolendrę, bazylię i~odrobinę soli.
Dokładnie wymieszać u~uformować w~kotlety. Grillować lub smażyć na oleju. Można
także je upiec w~piekarniku na papierze do pieczenia (\qty{180}{\celsius} przez
\qtyrange{20}{25}{\minute}). Podawać z~bułką. Nakładać w~następującej
kolejności: dolna część bułki, ketchup, ser, pomidor, kotlet, ogórek, sałata,
krążek cebuli, górną część bułki.

\recipe{Jajecznica z~szynką -- śniadanie}

\begin{Ingred}
    \item \qty{100}{\gram} jajek
    \item \qty{10}{\gram} szczypiorku
    \item \qty{70}{\gram} chleba żytniego
    \item \qty{1}{\gram} pieprzu
    \item \qty{1}{\gram} soli
    \item \qty{15}{\gram} masła
    \item \qty{45}{\gram} szynki z~piersi kurczaka
\end{Ingred}

Jajka wbić do miseczki. Rozgrzać patelnię i~na małym ogniu roztopić masło,
dodać jajka i~oprószyć delikatnie solą. Smażyć na małym ogniu, delikatnie
przesuwając jajka drewnianą łyżką. Pod koniec smażenia jajecznicy dodać
wędlinę, doprawić świeżo zmielonym czarnym pieprzem. Po nałożeniu do naczyń,
jajecznicę posypać szczypiorkiem.

\recipe{Pudding chia z~musem malinowym -- \rom{2} śniadanie}

\begin{Ingred}
    \item \qty{50}{\gram} borówki amerykańskiej
    \item \qty{150}{\gram} malin
    \item \qty{180}{\gram} jogurtu naturalnego \qty{2}{\percent}
    \item \qty{30}{\gram} nasion chia
    \item \qty{7}{\gram} ksylitolu
\end{Ingred}

Do jogurtu dodać nasiona chia. Wymieszać i~odstawić. Maliny zmiksować wraz
z~ksylitolem. Polać na wierzch puddingu chia. Włożyć pudding na całą noc do
lodówki. Posypać borówką amerykańską.

\recipe{Koszotto meksykańskie -- obiad}

\begin{Ingred}
    \item \qty{100}{\milli\litre} bulionu warzywnego
    \item \qty{50}{\gram} cebuli
    \item \qty{10}{\milli\litre} oleju rzepakowego
    \item \qty{200}{\gram} pomidorów w~puszce
    \item \qty{1}{\gram} suszonego oregano
    \item \qty{1}{\gram} pieprzu czarnego
    \item \qty{1}{\gram} soli
    \item \qty{50}{\gram} kaszy bulgur
    \item \qty{60}{\gram} konserwowej kukurydzy
    \item \qty{80}{\gram} konserwowej fasoli czerwonej
\end{Ingred}

Na rozgrzanej patelni na oleju zeszklić pokrojoną w~kosteczkę cebulę. Dodać
kaszę i~smażyć mieszając przez około minutę. Dodać gorący bulion oraz suszoną
paprykę wędzoną jeśli jej używamy, przykryć i~gotować przez \qty{10}{\minute}.
Dodać pomidory, kukurydzę, fasolę wraz z~zalewą, wymieszać i~zagotować. Gotować
bez przykrycia przez około \qty{15}{\minute} (w przypadku świeżych pomidorów na
większym ogniu), aż kasza będzie miękka, napęcznieje i~powstanie gęsty sos. W
trakcie gotowania kilka razy danie przemieszać. W przypadku gęstych pomidorów
z~puszki można dodawać więcej bulionu. Można użyć passaty pomidorowej. Na
wierzch można posypać kolendrą.

\recipe{Sałatka z~arbuzem, serem feta, oliwkami i~cebulką -- kolacja}

\begin{Ingred}
    \item \qty{350}{\gram} arbuza
    \item \qty{50}{\gram} czerwonej cebuli
    \item \qty{10}{\milli\litre} oliwy z~oliwek
    \item \qty{1}{\gram} pieprzu
    \item \qty{40}{\gram} rukoli
    \item \qty{80}{\gram} sera feta
    \item \qty{1}{\gram} soli
    \item \qty{12}{\milli\litre} soku z~cytryny
    \item \qty{12}{\gram} miodu
    \item \qty{10}{\gram} oliwek czarnych
\end{Ingred}

Do salaterki włożyć rukolę, dodać arbuza, fetę, pokrojone na plasterki oliwki,
cienko posiekaną czerwoną cebulą. Posypać miętą. Wymieszać składniki winegretu
i~poleć po sałatce. Posypać świeżo mielonym pieprzem.

\recipe{Czekoladowy jogurt -- śniadanie}

\begin{Ingred}
    \item \qty{10}{\gram} kakao \qty{16}{\percent}
    \item \qty{120}{\gram} malin
    \item \qty{7}{\gram} migdałów
    \item \qty{20}{\gram} płatków owsianych
    \item \qty{1}{\gram} mielonego cynamonu
    \item \qty{10}{\gram} nasion chia
    \item \qty{180}{\gram} jogurtu naturalnego \qty{2}{\percent}
\end{Ingred}

Jogurt wymieszać z~kakao. Dodać płatki owsiane, maliny, wszystko posypać
płatkami migdałów.

\recipe{Zapiekani pełnoziarniste -- \rom{2} śniadanie}

\begin{Ingred}
    \item \qty{90}{\gram} bułki grahamki
    \item \qty{1}{\gram} pieprzu czarnego
    \item \qty{20}{\gram} rukoli
    \item \qty{1}{\gram} soli
    \item \qty{30}{\gram} szynki z~piersi kurczaka
    \item \qty{30}{\gram} sera goudy
    \item \qty{15}{\gram} oliwek zielonych
    \item \qty{40}{\gram} konserwowej kukurydzy
    \item \qty{15}{\gram} ketchupu
\end{Ingred}

Bułkę posmarować ketchupem. Dodać kukurydzę, wędlinę, oliwki oraz ser. Można
także opcjonalnie dodać pieczarki. Piec w~piekarniku nagrzanym do
\qty{160}{\celsius} przez \qty{10}{\minute}. Na wierzch dodać rukolę.

\recipe{Smażony ryż z~kurczakiem i~warzywami -- obiad}

\begin{Ingred}
    \item \qty{50}{\gram} jajek
    \item \qty{100}{\gram} mięsa z~piersi kurczaka
    \item \qty{100}{\gram} brokułów
    \item \qty{45}{\gram} marchwi
    \item \qty{10}{\milli\litre} oleju rzepakowego
    \item \qty{60}{\gram} pieczarki
    \item \qty{1}{\gram} pieprzu czarnego
    \item \qty{50}{\gram} ryżu brązowego
    \item \qty{10}{\milli\litre} sosu sojowego
    \item \qty{1}{\gram} soli
    \item \qty{100}{\gram} fasoli szparagowej
\end{Ingred}

Obgotować krótko brokuła, fasolkę i~marchewkę. Ryż ugotować w~osolonej wodzie.
Mięso z~kurczaka oczyścić i~pokroić w~kostkę. Oprószyć solą i~pieprzem. Na
patelni rozgrzać olej, obsmażyć kurczaka ze wszystkich stron. Dodać pokrojone
w~plasterki pieczarki. Smażyć chwilę, po czym dodać brokuła. Po kolejnych
\qtyrange{2}{3}{\minute} dodać fasolkę i~marchewkę, pokrojoną w~cienkie słupki.
Dolać sos sojowy. Po chwili wbić jajko i~szybko je wymieszać, aby pokryło
warzywa. Smażyć przez \qtyrange{2}{3}{\minute}. Dodać ryż i~smażyć kolejne
\qtyrange{2}{3}{\minute}, po czym przełożyć na talerz.

\recipe{Zupa krem z~soku pomidorowego z~cukinią -- podwieczorek}

\begin{Ingred}
    \item \qty{2}{\gram} bazylii
    \item \qty{250}{\milli\litre} bulionu warzywnego
    \item \qty{50}{\gram} cebuli
    \item \qty{125}{\gram} cukinii
    \item \qty{10}{\milli\litre} oliwy z~oliwek
    \item \qty{1}{\gram} pieprzu czarnego
    \item \qty{1}{\gram} soli
    \item \qty{250}{\milli\litre} soku pomidorowego
\end{Ingred}

Cukinie umyć, obrać ze skórki i~pokroić na mniejsze kawałki. W garnku rozgrzać
oliwę, zeszklić drobno pokrojoną cebulę. Dodać pokrojoną cukinię, smażyć krótką
chwilę. Wlać bulion warzywny, gotować około \qty{10}{\minute} do miękkości
cukinii, zblendować. Dodać sok pomidorowy, świeżą bazylię, ponownie zblendować
na krem. Doprawić do smaku solą, pieprzem.

\recipe{Omlet -- kolacja}

\begin{Ingred}
    \item \qty{100}{\gram} jajek
    \item \qty{50}{\gram} cebuli
    \item \qty{5}{\gram} szczypiorku
    \item \qty{10}{\milli\litre} oliwy z~oliwek
    \item \qty{170}{\gram} pomidorów
    \item \qty{1}{\gram} pieprzu czarnego
    \item \qty{1}{\gram} soli
    \item \qty{25}{\gram} szpinaku
\end{Ingred}

Warzywa umyć i~osuszyć. Pomidory sparzyć wrzątkiem, obrać ze skórki i~pokroić
w~małą kostkę. Na rozgrzanej oliwie z~oliwek zeszklić pokrojoną w~drobną kostkę
cebulę, dodać pomidor, przyprawić solą morską i~pieprzem, poddusić. Jajka
roztrzepać, przyprawić pieprzem, dodać do nich podduszone warzywa. W naczyniu
żaroodpornym ułożyć umyte i~osuszone liście szpinaku, wlać przygotowaną masę
jajeczną z~pomidorami. Piec pod przykryciem około \qty{20}{\minute}
w~temperaturze \qty{160}{\celsius}. Gotowy omlet posypać posiekanym
szczypiorkiem. Lub przygotować omlet tradycyjnie na patelni.

\end{document}
