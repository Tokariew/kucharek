\documentclass[main.tex]{subfiles}
\begin{document}

\recipe{Gulasz Wiejski}

\begin{Ingred}
    \item \qty{700}{\gram} wołowiny
    \item \num{5} małych cebul
    \item \qty{200}{\gram} białej cebuli
    \item \num{3} ziemniaki
    \item \num{2} łyżki koncentratu pomidorowego
    \item \num{4} łyżki śmietany \qty{22}{\percent}
    \item \num{2} łyżki przyprawy do gulaszu
    \item marchewka
    \item sól
    \item olej do smażenia
\end{Ingred}

Wołowinę pokroić w~kostkę i~obsmażyć na gorącym oleju. Zalać taką ilością wody,
aby mięso było przykryte. Dodać przyprawę i~gotować na małym ogniu do miękkości
mięsa około \qtyrange{1}{1.5}{\hour}. Cebulę pokroić na \num{4} części,
marchewkę w~plasterki, ziemniaki w~kostkę, a~kapustę poszatkować. Cebulę
i~marchewkę podsmażyć na niewielkiej ilości oleju i~razem z~ziemniakami
i~kapustą dodać do mięsa. Dusić do miękkości warzyw. Dodać koncentrat
pomidorowy, przyprawić do smaku solą. Podawać w~miseczkach, przybrać śmietaną.

\recipe{Zupa fasolowa}

\begin{Ingred}
    \item \num{3} ziemniaki (\qty{40}{\deca\gram})
    \item \qty{20}{\deca\gram} białej fasoli perłowej
    \item \num{3} pomidory (\qty{25}{\deca\gram})
    \item \qty{10}{\deca\gram} ryżu
    \item \num{1} ząbek czosnku
    \item \num{2} łyżki masła
    \item \num{2} łyżeczki ziół prowansalskich
    \item \qty{10}{\deca\gram} startego żółtego sera
    \item szczypta \enquote{Kucharka}
\end{Ingred}

Białą fasolę namoczyć przez noc. Ziemniaki obrać ze skórki i~pokroić w~kostkę.
Pomidory obrać ze skórki i~pokroić w~ćwiartki. Ziemniaki i~fasolę zalać
6~szklankami wody, gotować \qty{20}{\minute}. Po tym czasie dodać opłukany ryż
oraz pomidory, doprawić przyprawą „Kucharek” i~gotować następne
\qty{20}{\minute}. Zmiażdżony czosnek utrzeć z~masłem oraz ziołami
prowansalskimi i~włożyć do wazy. Ser wrzucić do gorącej zupy, przyprawić zupę
do smaku, wymieszać, wlać do wazy.

\recipe{Strogonow}

\begin{Ingred}
    \item \qty{1}{\kilo\gram} wołowiny w~paskach
    \item \qty{1}{\kilo\gram} pieczarek
    \item \num{2} cebule
    \item koncentrat pomidorowy
    \item śmietana
    \item mąka
\end{Ingred}

\recipe{Zupa pieczarkowa}

\begin{Ingred}
    \item \qty{550}{\gram} pieczarek
    \item kostka wołowa
    \item margaryna
    \item \num{1.5} szklanki śmietany \qty{30}{\percent}
    \item trochę białego wina
    \item sól
    \item pieprz
    \item gałka muszkatołowa
\end{Ingred}

Pieczarki posiekać, margarynę rozpuścić, dodać kostkę wołową, pieczarki
i~śmietanę. Dolać białego wina, doprawić.

\recipe{Zupa z~suszonymi pomidorami i~mascarpone}

\begin{Ingred}
    \item \qty{2}{\litre} bulionu drobiowego lub warzywnego
    \item \num{1} średnia cebula
    \item \num{1} mały por
    \item \num{2} ząbki czosnku
    \item \qty{150}{\gram} suszonych pomidorów
    \item \num{1} puszka pomidorów
    \item \num{.5} łyżeczki oregano
    \item \num{.5} łyżeczki bazylii
    \item \num{2} łyżki masła
    \item \num{4} łyżki mascarpone
    \item sól
    \item pieprz
\end{Ingred}

Pora pokroić w~drobne krążki, cebulę w~kosteczkę. Na patelni rozpuścić masło,
wrzucić cebulę i~pora, zeszklić je. Posiekany czosnek dodać na koniec smażenia
i~chwilę razem wszystko smażyć. Przełożyć do garnka. Na patelni następnie
podsmażyć pokrojone w~wąskie paseczki suszone pomidory, dodać pomidory
z~puszki, chwilę smażyć razem, później przełożyć do garnka. Zawartość garnka
zalać \qty{2}{\litre} bulionu. Dusić całość około \qty{20}{\minute}. Zmiksować
na gładki krem, doprawić solą, pieprzem i~ziołami. Na koniec dodać mascarpone,
wymieszać, aż się dokładnie rozpuści.

\recipe{Zupa z~dynią}

\begin{Ingred}
    \item \qty{1}{\kilo\gram} dyni
    \item \qty{.75}{\litre} bulionu
    \item \num{1} duża cebula pokrojona w~kostkę
    \item \num{1} łyżka masła
    \item \numrange{3}{4} marchewki
    \item \num{1} łyżka przecieru pomidorowego
    \item \num{1} szklanka makaronu imitującego ryż
    \item \qty{100}{\gram} serka topionego
    \item \num{4} łyżki gęstej śmietany
    \item \num{.5} łyżeczki curry
    \item \num{.5} łyżeczki chilli
    \item \num{.5} łyżeczki mielonego kminku
    \item sól
    \item cukier
    \item pieprz
\end{Ingred}

Obrać dynię, pokroić w~kostkę o~boku \qty{1}{\centi\metre}. Rozpuścić masło,
dodać do tego cebulę i~dusić przez \qty{12}{\minute}. Dodać \qty{.25}{\litre}
bulionu. Marchewkę przecierać na najdrobniejszych oczkach na miazgę. Rozdrobnić
dynię po wystygnięciu, dodać marchewkę i~pozostałą część bulionu, zagotować
przez \qty{5}{\minute}. Po pewnym czasie dorzucić serek, przecier rozmieszany
ze śmietaną oraz przyprawy. Gotować \qtyrange{3}{4}{\minute}.

\recipe{Mleczna zupa dyniowa}

\begin{Ingred}
    \item \qty{1}{\kilo\gram} dyni
    \item \num{1} cebula cukrowa
    \item \qty{3}{\centi\metre} kawałek imbiru
    \item \qty{.5}{\litre} mleka
    \item sól
    \item pieprz
    \item cukier
    \item masło
\end{Ingred}

Dynię pokroić w~kostkę, wrzucić do garnka, zalać odrobiną wody i~gotować, aż
się lekko rozpadnie. Na patelni rozpuścić masło i~dusić cebulę pokrojoną
w~talarki. Uduszoną cebulę dodać do dyni, zetrzeć na drobnej tarce imbir
i~dodać. Kiedy dynia jest już ugotowana, zmiksować całość blenderem. Dolewać
stopniowo mleko, podgrzewać dalej, doprawić do smaku. Jeżeli zupa jest zbyt
pikantna, dolać więcej mleka, można podawać z~listkami mięty.

\recipe{Zupa krem dyniowa}

\begin{Ingred}
    \item \qty{500}{\gram} surowej dyni
    \item \num{2} ziemniaki
    \item \num{1} mała cebula pokrojona w~kostkę
    \item \num{1} posiekany ząbek czosnku
    \item \qty{1}{\litre} rosołu lub wywaru z~warzyw
    \item \num{1} łyżeczka curry
    \item \num{4} łyżki oliwy
    \item sól
    \item pieprz
    \item cukier
\end{Ingred}

Na oliwie zeszklić cebulę i~czosnek. Dodać dynię i~ziemniaki pokrojone
w~kostkę. Lekko obsmażyć, dodać curry, wymieszać. Po \qty{5}{\minute} wlać
wywar i~gotować do momentu, aż dynia i~ziemniaki będą miękkie, całość zmiksować
i~przyprawić.

\recipe{Zupa z~dyni „serowa”}

\begin{Ingred}
    \item \qty{1}{\kilo\gram} obranej i~pokrojonej dyni
    \item \qty{2}{\litre} rosołu lub bulionu
    \item \num{5} czubatych łyżek tartego sera
    \item \num{5} czubatych łyżek sera tupu Gruyere lub innego ostrego sera
    \item \num{2} cebule średniej wielkości
    \item \num{1} czubata łyżka masła
    \item sól
    \item pieprz czarny mielony
\end{Ingred}

Na maśle podsmażyć cebulę pokrojoną w~kostkę. Do rosołu włożyć obraną i~pociętą
na kawałki dynię, cebulę, troszkę soli i~pieprzu. Całość gotować około
\qty{30}{\minute} pod przykryciem. Następnie zmiksować zupę. Zdjąć z~gazu,
wsypać 5 czubatych łyżek drobno startego ostrego sera. Zupy nie podawać gorącej
ani chłodnej.

\end{document}
