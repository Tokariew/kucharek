\documentclass[../main.tex]{subfiles}
\begin{document}

\recipe{Ogórki korniszona}

\begin{Ingred}[zalewa]
    \item \numrange{5}{6} szklanek wody
    \item \num{1} szklanka octu \qty{10}{\percent}
    \item \num{2} łyżki soli
    \item \num{1} szklanka cukru
    \item koper
    \item chrzan
    \item gorczyce
    \item ziele angielskie
    \item pieprz ziarnisty
    \item liście laurowe
    \item czosnek
    \item marchew
    \item cebule
\end{Ingred}

Do słoików wkładamy gałązkę kopru, dwa krążki chrzanu, po dwa ziarenka pieprzu
i ziela, jeden listek laurowy, ząbek czosnku, pięć krążków marchewki, dwa
krążki cebuli.

Do tak przygotowanych słoików szczelnie układamy ogórki, zalewę gotujemy,
gorącą zalewamy słoiki, zakręcamy. Pasteryzujemy około
\qtyrange{5}{7}{\minute}.

Do konserwowania wybieram dosyć mała ogóreczki, koniecznie z~jasnym końcem
z~jednej strony, trochę \enquote{dziobate}. Następnie myjemy i~zalewamy zimną
wodą, aby się trochę pomęczyły, kilka godzin lub całą noc. Tak przygotowane
ogórki układamy w~słoikach pionowo, dodając do każdego słoja wymieszane
w~składnikach obok przyprawy. Gotujemy wszystkie składniki zalewy i~po
ostudzeniu zalewamy ogórki. Pasteryzujemy kilka minut. Wyciągamy dokręcamy
wieczka i~odwracamy.

Słoiki do pasteryzacji do wody o~takiej mniej więcej temperatury jaką ma
zalewa. Na blachę wlewamy wodę około do \qty{2}{\centi\metre} wysokości,
piekarniki ustawiamy na \qtyrange{130}{140}{\celsius} grzane jedynie od dołu.
Wkładamy słoiki i~ogrzewamy do czasu jak ogórki zmienią kolor, słoiki pokryją
się rosą, a~wieczka ich są już gorące.

\recipe{Sałatka z~ogórków od Babci}

\begin{Ingred}
    \item \qty{4}{\kilo\gram} obranych i~pokrojonych ogórków
    \item \qty{0.5}{\kilo\gram} podgotowanych plasterków marchewki
    \item \qty{0.5}{\kilo\gram} cebuli
    \item \qty{0.5}{\kilo\gram} papryki czerwonej
\end{Ingred}

\begin{Ingred}[zalewa]
    \item \num{1} szklanka cukru
    \item \num{0.5} szklanki oleju
    \item \num{1} szklanka octu
    \item \num{1} łyżeczka pieprzu mielonego
    \item po \num{1} ząbku czosnku do słoików
\end{Ingred}

Zostawić zasypane przyprawami na \qtyrange{1}{2}{\hour}. Po tym czasie warzywa
puszczą sok. Wkładać do słoików razem z~sokiem z~warzyw. Pasteryzować około
\qty{15}{\minute}.

\recipe{Ogórki z~czosnkiem}

\begin{Ingred}
    \item \qty{4}{\kilo\gram} ogórków obranych
    \item \num{6} ząbków czosnku lub więcej
    \item \num{1} szklanka cukru.
    \item niepełna \num{1} szklanka octu \qty{10}{\percent}
    \item \num{1} szklanka oleju
    \item \num{4} łyżka soli
\end{Ingred}

Ogórki pokroić w~cieniutkie plasterki. Dodać pozostałe składniki i~wymieszać.
Odstawić na \qtyrange{1}{2}{\hour} Wkładać do wyparzonych słoików
i~pasteryzować \qtyrange{10}{15}{\minute}.

\recipe{Sałatka na zimę}

\begin{Ingred}
    \item \qty{1}{\kilo\gram} marchwi
    \item \qty{1}{\kilo\gram} kapusty białej
    \item \qty{1}{\kilo\gram} cebuli
    \item \qty{1}{\kilo\gram} papryki
    \item \qty{1}{\kilo\gram} pomidorów
\end{Ingred}

\begin{Ingred}[zalewa]
    \item \num{1} szklanka wody
    \item \num{1} szklanka oleju
    \item \num{1} szklanka octu
    \item \num{0.75} szklanki cukru
    \item \num{3} łyżki soli
    \item liść laurowy
    \item pieprz ziarnisty
    \item ziele angielskie
\end{Ingred}

Marchew zatrzeć na tarce jarzynowej o~grubych oczkach, kapustę poszatkować,
cebulę pokroić w~półtalarki, paprykę w~cienkie paseczki, pomidory na mniejsze
ćwiarteczki. Zagotować zalewę i~wrzucać po kolei warzywa marchew gotując
\qty{5}{\minute} do marchwi dorzucić kapustę gotować razem \qty{5}{\minute}
następnie dorzucić cebule gotować razem kolejne \qty{5}{\minute}. Następnie
dodać paprykę i~gotować kolejne \qty{5}{\minute} a~na końcu pomidory i~znów
razem gotować \qty{5}{\minute}. Gorącą sałatkę wkładać do słoików i~układać na
ściereczce do góry dnem. Nie pasteryzować.

\recipe{Buraczki z~papryką na zimę}

\begin{Ingred}
    \item \qty{4}{\kilo\gram} ugotowanych buraczków
    \item \qty{1}{\kilo\gram} cebuli pokrojonej na talarki
    \item \num{3} papryki pokrojone w~kostkę lub cienkie plasterki
    \item \num{1} szklanka octu
    \item \num{1} szklanka oleju
    \item \num{1} szklanka cukru
\end{Ingred}

Buraczki zatrzeć na tarce, paprykę pokroić na wąskie paseczki lub w~małą
kosteczkę, cebulkę podsmażyć \num{0.5} szklanki oleju, wystudzić. Zagotować
ocet z~cukrem i~wlać w~sałatkę, dolać resztę oleju, wymieszać. Wkładać do
słoików i~pasteryzować je około \qty{20}{\minute}.

\recipe{Sos słodko–kwaśny na zimę}

\begin{Ingred}
    \item \qty{3}{\kilo\gram} pomidorów
    \item \qty{1}{\kilo\gram} cebuli
    \item \num{1} papryki czerwonej
    \item \num{1} puszka ananasów
    \item \num{1} puszka kukurydzy
    \item \num{0.5} łyżeczki chili
    \item \num{3} ząbki czosnku
    \item \num{2} łyżki musztardy
    \item \num{1} łyżka słodkiej papryki mielonej
    \item \num{1} łyżka Vegety
    \item \num{1} łyżka curry
    \item \num{0.5} łyżeczki mielonego pieprzu
    \item \num{3} szklanki cukru
    \item \numrange{0.5}{0.75} szklanki octu
    \item \num{2} łyżki soli
    \item \num{2} czubate łyżki mąki ziemniaczanej
\end{Ingred}

Pomidory obrać ze skórki, pokroić w~kostkę. Cebulę pokroić w~kostkę i~razem
z~pomidorami, posolić, gotować na wolnym ogniu, około godziny. Następnie dodać
pokrojonego w~kostkę ananasa, pokrojoną paprykę i~pozostałe składniki, oprócz
mąki i~gotować na wolnym ogniu około /num{0.5} godziny, albo na tyle długo, aby
papryka się nie rozgotowała. Mąkę rozrobić w~soku z~ananasa i~na końcu dodać do
sosu. Zagotować. Gorący sos wkładać do słoików. Odwrócić do góry dnem
i~pozostawić do ostudzenia. Nie pasteryzować.

\recipe{Sos słodko–kwaśny wersja \rom{2}}

\begin{Ingred}
    \item \qty{3}{\kilo\gram} ogórków
    \item \qty{1}{\kilo\gram} cebuli pokrojonej w~kostkę
    \item \num{2} łyżki soli
    \item \num{1} paprykę pokrojoną w~kostkę
    \item \num{2.5} szklanki cukru
    \item \num{0.75} szklanki octu
    \item \num{2} łyżki papryki słodkiej
    \item \num{0.25} łyżki chili
    \item \num{1} łyżka curry
    \item \numrange{0.25}{0.5} łyżki pieprzu białego
    \item \num{1} łyżka czosnku mielonego
    \item \num{1} puszka kukurydzy
    \item \num{1} puszka ananasa
    \item \num{2} łyżki musztardy
\end{Ingred}

Pomidory i~cebuli gotować przez \qty{1}{\hour}. Dodać pozostałe składniki
i~gotować wszystko przez kolejne \qty{0.5}{\hour}. Na koniec dodać \num{2}
łyżki mąki ziemniaczanej rozmieszanej w~sporej ilości sosu. Dodajemy do garnka,
zagotowujemy i~gorący sos nakładamy do słoików.

\recipe{Ogórki chili}

\begin{Ingred}
    \item \qty{2}{\kilo\gram} ogórków
    \item \num{2} łyżki soli
    \item \num{1} duża główka czosnku
\end{Ingred}

\begin{Ingred}[zalewa]
    \item \num{1} szklanka octu
    \item \numrange{0.5}{1} szklanka wody
    \item \num{0.75} szklanki cukru
    \item \numrange{2}{3} łyżeczki chili w~proszku
    \item \num{3} łyżki oleju
\end{Ingred}

Ogórki umyć, poszatkować lub pokroić w~plasterki, zasypać \num{2} czubatymi
łyżkami soli, zostawić na \qty{4}{\hour}, aby puściły wodę. Odlać wodę, dodać
przeciśnięty przez praskę czosnek, wsypać chili, wlać olej i~dokładnie
wymieszać. Przełożyć do słoików.

Przygotować zalewę. Ocet zagotować z~cukrem i~wodą. Gorącą zalewą zalać ogórki,
nie trzeba pasteryzować.

\recipe{Ogórki chili \rom{2}}

\begin{Ingred}
    \item \qty{2.5}{\kilo\gram} ogórków
    \item \num{4} łyżki soli
    \item \num{2} płaskie łyżki chili
    \item \num{1} główka czosnku
    \item \num{7} łyżek oleju
    \item \num{1.5} szklanki octu
    \item \qty{60}{\deka\gram} cukru
\end{Ingred}

Ogórki pokroić na \num{4} części wzdłuż, zasypać solą. Odstawić na
\qty{6}{\hour}. Odlać wodę. Dodać chili, wciśnięty lub drobno pokrojony
czosnek, olej. Wszystko wymieszać. Przygotować zalewę, ocet zagotowany
z~cukrem. Gorącą zalewę wlać do ogórków. Wymieszać, odstawić na
\qty{12}{\hour}. Nakładać do słoików, dość ściśle, zalewa powinna w~całości
przykrywać ogórki. Nie gotować.

\recipe{Grzyby marynowane}

Grzyby, kapelusze lub małe grzybki z~trzonami, gotować przez \qty{10}{\minute}
z~dużą ilością soli: na \qty{4}{\litre} garnek wypełniony prawie po brzegi wodą
i~grzybami dać \num{3} kopiate łyżki soli. Taka ilość soli zapewnia, że grzyby
nie będą kwaśne.

Jako zalewę stosować \num{1} szklankę octu \qty{10}{/percent} na \num{4}
szklanki wody. Na te pięć szklanek zalewy dodać łyżeczkę soli i~\num{5}
kopiatych łyżeczek cukru. Dodatkowo parę listków laurowych \numrange{8}{10}
ziaren ziela angielskiego, kilkanaście ziaren pieprzu czarnego. Pasteryzować po
ostygnięciu zalewy.

\recipe{Sałatka z~cukinii na zimę}

\begin{Ingred}
    \item \qty{2}{\kilo\gram} cukinii
    \item \qty{250}{\gram} cebuli
    \item \num{2} czerwone papryki
    \item \qty{250}{\gram} marchewki
    \item \num{3} łyżki soli
    \item \numrange{2}{3} ząbki czosnku
    \item \numrange{1}{2} papryczek chili, bez pestek
    \item \num{3} łyżeczki gorczycy
    \item \num{4} liście laurowe
    \item \num{1} łyżeczka ziela angielskiego
    \item kilka ziarenek kolorowego pieprzu
    \item posiekany koperek lub nać pietruszki
\end{Ingred}

\begin{Ingred}[zalewa]
    \item \num{1.5} szklanki wody
    \item \num{0.75} szklanki octu \qty{10}{\percent}
    \item \num{0.5} szklanki cukru
    \item \num{4} łyżki oleju
\end{Ingred}

Przygotowanie sałatki z~cukinii najlepiej rozpocząć na około
\qtyrange{6}{8}{\hour} przed planowanym nakładaniem jej do słoików. Wszystkie
warzywa dokładnie myjemy, osuszamy. Najlepiej wybrać niezbyt duże
i~nieprzerośnięte cukinie o~delikatnej skórce.

\end{document}
