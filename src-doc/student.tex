\documentclass[../main.tex]{subfiles}
\begin{document}

\recipe{Kremowa kasza manna na mleku}

\begin{Ingred}
    \item \num{1.5} szklanki mleka
    \item \num{0.5} szklanki śmietany kremówki \qty{30}{\percent}
    \item \num{1} łyżka cukru
    \item \num{2.5} łyżki kaszy manny
    \item szczypta soli
\end{Ingred}

Szklankę mleka gotujemy z~kremówką, szczyptą soli i~cukrem w~garnuszku z~grubym
dnem. W szklance rozdrabniamy pozostałe pół szklanki zimnego mleka i~kaszę.
Rozdrobnioną kaszę wlewamy do gotującego się mleka i~doprowadzamy do wrzenia.
Gotujemy \qtyrange{2}{3}{\minute}. Podajemy ciepłą lub przestudzoną z~owocami.

\recipe{Remedium na kaca}

\begin{Ingred}
    \item \num{1} żółtko
    \item \num{1} łyżka cukru
    \item \num{3} łyżki śmietany
    \item \num{1} białko
    \item \num{4} łyżki wódki \qty{40}{\percent}
    \item gałka muszkatołowa
\end{Ingred}

Napój przygotowujemy przed spożyciem alkoholu. Miksujemy żółtko z~cukrem,
śmietaną i~gałką muszkatołową i~dwiema łyżkami wódki. Odstawiamy do lodówki na
\qty{0.5}{\hour} po czym dodajemy kolejne dwie łyżki wódki i~ubitą pianą
z~białka. Przechowujemy w~lodówce i~wypijamy, kiedy dopadnie nas kac.

\recipe{Żółty ryż ze smażony z~jajkiem}

\begin{Ingred}
    \item \num{1} szklanka ryżu basmati
    \item \num{2} jajka
    \item \num{1} ząbek czosnku przeciśnięty przez praskę
    \item \num{1} łyżeczka startego imbiru
    \item \num{2} łyżki kurkumy
    \item \num{3} łyżki sosu sojowego
    \item \num{1} łyżka soku z~cytryny
    \item szczypta soli
\end{Ingred}

Ryż gotujemy według przepisu. Po wystygnięciu ryżu, na rozgrzaną patelnie
wrzucamy czosnek, imbir, dodajemy kurkumę i~zalewamy sosem sojowym. Dodajemy
dwa całe jajka i~mieszamy. Gdy jajka się lekko zetną, dodajemy ryż i~dokładnie
mieszamy, aż całość będzie miała ładny, żółty kolor. Na koniec dodajemy
szczyptę soli i~sok z~cytryny, mieszamy. Zdejmujemy z~ognia. Podawać gorący, na
przykład z~kurczakiem.

\recipe{Pałki kurczaka w~coli}

\begin{Ingred}
    \item \num{10} pałek z~kurczaka
    \item \num{0.5} szklanki ciemnego sosu sojowego
    \item \num{1} łyżka octu
    \item \num{2} szklanki coli
    \item imbir
    \item czarny pieprz
    \item pieprz cayenne
    \item kilka kropel tabasco
    \item olej do smażenia
\end{Ingred}

Pałki myjemy, osuszamy ręcznikiem papierowym. Robimy marynatę mieszając sos
sojowy, ocet, tabasco raz przyprawy: imbir, czarny pieprz i~pieprz cayenne.
Kurczaka umieszczamy w~marynacie tak, aby pokryła całe mięso i~zostawiamy na
\qtyrange{2}{3}{\hour}. Na rozgrzanym oleju podsmażamy kurczaka na złoto ze
wszystkich stron. Polewamy colą i~zmniejszamy ogień. Dusimy pod przykryciem
około \qty{40}{\minute} aż mięso będzie mięciutkie, a~sos zgęstnieje.

\recipe{Sałatka z~grillowanych warzyw}

\begin{Ingred}
    \item papryka
    \item cukinia
    \item pieczarki
    \item cebula
    \item suszone pomidory
    \item przyprawy
\end{Ingred}

\begin{Ingred}[sos]
    \item \num{2} łyżki oliwy z~oliwek
    \item \num{3} łyżki wody mineralnej
    \item \num{1} łyżka musztardy
    \item \num{1} ząbek czosnku
    \item sok z~\num{0.5} cytryny
    \item oregano
    \item bazylia
    \item sól
    \item pieprz
\end{Ingred}

Paprykę pokroić w~paski, cukinię i~cebulę pokroić na plasterki. Pieczarki
pokroić w~ćwiartki lub połówki, w~zależności od wielkości pieczarek. Cukinię
posypać solą i~odstawić na \qty{10}{\minute}, następnie opłukujemy ją wodą.
Wszystkie warzywa grillujemy. Przygotować sos — wszystkie składniki umieścić
w~słoiczku, zakręcić i~wymieszać. Warzywa wrzucić do miseczki, dodać suszone
pomidory, polać sosem.

\recipe{Chilli con carne}

\begin{Ingred}
    \item \qty{300}{\gram} mielonego mięsa wieprzowego
    \item \num{1} puszka krojonych pomidorów
    \item \num{1} puszka kukurydzy
    \item \num{1} puszka czerwonej fasoli
    \item \num{1} czerwona papryka
    \item \num{1} czerwona cebula
    \item \num{1} papryczka chilli
    \item jogurt naturalny
    \item posiekany szczypiorek
    \item przyprawa chilli w~proszku
    \item szczypta cynamonu
    \item szczypta gałki muszkatołowej
\end{Ingred}

Na oliwie z~oliwek podsmażamy pokrojoną w~kostkę cebulkę. Następnie dodajemy
pokrojoną paprykę oraz pokrojoną i~pozbawianą pestek papryczkę chilli. Po około
\qty{3}{\minute} dodajemy mięso, smażymy wszystko przez około \qty{5}{\minute}.
Dodajemy pomidory z~puszki, fasolę i~kukurydzę. Gotujemy na średnim ogniu przez
około \qty{10}{\minute}. Doprawiamy chilli, cynamonem i~gałką muszkatołową oraz
solą i~pieprzem do smaku. Podać z~łyżką jogurtu na wierzchu i~posypać
szczypiorkiem.

\recipe{Kurczak z~orzeszkami}

\begin{Ingred}
    \item \num{3} łyżki niesolonych orzeszków ziemnych
    \item \num{2} piersi kurczaka pokrojone w~kostkę
\end{Ingred}

\begin{Ingred}[marynata]
    \item \num{1} białko
    \item \num{1} łyżka mąki ziemniaczanej
    \item \num{1} łyżeczka utartego korzenia imbiru
    \item \num{0.5} pokrojonego w~skośne plastry pora
    \item \num{2} posiekane ząbki czosnku
    \item szczypta kruszonego chilli
\end{Ingred}

\begin{Ingred}[sos]
    \item \num{1} łyżki miodu
    \item \num{1} łyżka octu ryżowego
    \item \num{1} łyżka wytrawnego sherry
    \item \num{2} łyżki sosu sojowego
    \item \num{4} łyżki oleju
\end{Ingred}

Marynatę wymieszać z~kurczakiem i~wstawić na \qty{15}{\minute} do lodówki.
Rozgrzać patelnie, dodać olej. Gdy olej będzie gorący, wrzucić pora, chilli
i~czosnek, mieszać. Po chwili dodać kurczaka, smażyć i~mieszać. Wlać wcześniej
przyrządzony sos – powinien trochę odparować i~przykryć kurczaka, na końcu
dodać orzeszki. Podawać z~ryżem.

\recipe{Racuchy na kefirze z~jabłkami}

\begin{Ingred}
    \item \num{.25} szklanki cukru
    \item \num{2} szklanki mąki
    \item \num{0.5} łyżeczki sody oczyszczonej
    \item \qty{500}{\milli\litre} kefiru
    \item \num{2} jajka
    \item jabłka
    \item cukier puder do posypania
\end{Ingred}

Kefir, jajka, sodę oczyszczoną, sól i~cukier dokładnie wymieszać. Następnie
stopniowo dodawać mąkę, wciąż mieszając do pozbycia się grudek. Jabłka pokroić
na cienkie plastry. Rozgrzać patelnię z~olejem, a~następnie nakładać na nią
łyżką porcję ciasta, na każdej położyć plasterek jabłka. Smażyć z~obydwu stron,
odsączyć z~nadmiaru tłuszczu na papierowym ręczniku. Posypać cukrem pudrem.

\recipe{Skrzydełka w~sosie sojowym z~miodem}

\begin{Ingred}
    \item \num{2} łyżki sosu sojowego
    \item \num{1} łyżka miodu
    \item \num{1} łyżka wytrawnego białego wina
    \item skrzydełka lub podudzia z~kurczaka
    \item rozgnieciony ząbek czosnku
    \item olej
    \item sól
    \item pieprz
\end{Ingred}

Skrzydełka smażyć na oleju, dodać czosnek i~przyprawy. Kiedy się zrumienią,
polewać mieszanką sosu sojowego, miodu i~wina. Dusić, aż nadmiar wody odparuje.
Podawać z~ryżem i~surówką.

\recipe{Zielony sos kokosowy z~kurczakiem}

\begin{Ingred}
    \item \qty{600}{\gram} piersi z~kurczaka
    \item \qty{200}{\gram} mleka kokosowego
    \item \qty{400}{\gram} brokułów
    \item \num{2} cebule
    \item \numrange{3}{4} ząbki czosnku
    \item \qty{200}{\milli\litre} śmietany kremówki \qty{30}{\percent}
    \item \num{1} łyżeczka musztardy
    \item sól
    \item pieprz
    \item curry
    \item czosnek granulowany
    \item oliwa
\end{Ingred}

Cebulę i~czosnek pokroić w~kostkę. Wrzucić na zimną oliwę i~zacząć smażyć do
zarumienienia. Wrzuć pokrojoną pierś z~kurczaka. Doprawić solą, pieprzem
i~czosnkiem granulowanym. W tym samym czasie gotować brokuły. Pokrojone dodać
do usmażonego kurczaka. Mieszać, dodać mleko kokosowe i~śmietanę. Doprawić
curry i~musztardą, gotować do zgęstnienia. Jeżeli jest zbyt rzadkie, zagęścić
łyżką mąki. Podawać z~ryżem lub makaronem.

\recipe{Zapiekanka z~ziemniakami}

\begin{Ingred}
    \item \qty{1}{\kilo\gram} ziemniaków
    \item \num{2} kiełbaski
    \item \qty{300}{\gram} boczku wędzonego
    \item \num{6} dużych cebul
    \item \num{1} puszka fasoli czerwonej konserwowanej
    \item \num{4} ząbki czosnku
    \item sól
    \item pieprz
    \item olej
    \item świeża pietruszka
\end{Ingred}

Rozgrzać na patelni \num{3} łyżki oleju, dodać pokrojony boczek i~pokrojoną
w~paseczki cebulę. Ziemniaki ugotować, ale aby nie były za miękkie. Po
pokrojeniu przysmażamy na drugiej patelni, wszystko razem łączymy, dodajemy
pokrojoną kiełbaskę, czosnek, sól, pieprz. Wymieszać, posypać świeżą
pietruszką.

\recipe{Kotlety z~kaszą i~pieczarkami}

\begin{Ingred}
    \item \qty{0.5}{\kilo\gram} mięsa mielonego
    \item \num{1} saszetka kaszy gryczanej
    \item \num{2} łyżki natki pietruszki
    \item \num{1} ząbek czosnku
    \item \num{4} pieczarki
    \item \num{0.5} cebuli
    \item sól
    \item pieprz
\end{Ingred}

Mięso mielone wymieszać z~wcześniej ugotowaną kaszą gryczaną. Dodać starte
pieczarki, startą cebulę i czosnek oraz przyprawić do smaku. Dodać natkę
pietruszki. Uformować kotlety i~obtoczyć je w~mące. Smażyć na oleju, aż się
zrumienią.

\end{document}
