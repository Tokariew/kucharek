\documentclass[../kucharek.tex]{subfiles}
\begin{document}

\recipe{Meksykański chłodnik z~awokado i~kolendrą}
\begin{Ingred}
    \item \num{2} duże dojrzałe awokado
    \item kilkanaście kukurydzianych nachos
    \item \num{2} szklanki rosołu z~kurczaka
    \item \num{.5} szklanki słodkiej śmietanki
    \item sól
    \item pieprz czarny
    \item kolendra
    \item natka pietruszki
\end{Ingred}

Awokado obrać, usunąć pestki. Miąższ zmiksować razem z~rosołem i~śmietanką.
Doprawić pieprzem, kolendrą i~ewentualnie solą. Schłodzić w~lodówce.

Przed podaniem nachos wstawić na \qty{5}{\minute} do gorącego piekarnika.
Podawać zimną zupę posypaną natką pietruszki z~gorącymi nachos do
przegryzienia.

\recipe{Tatar z~łososia z~kaparami i~szalotką}

\begin{Ingred}
    \item \qty{200}{\gram} filetu z~surowego łososia
    \item \qty{200}{\gram} filetu z~wędzonego łososia
    \item \num{4} szalotki
    \item \qty{10}{\gram} kaparów
    \item \num{4} łyżki oliwy z~oliwek
    \item \qty{50}{\milli\litre} wódki
    \item \num{2} łyżki koperku
    \item sok z~\num{.5} cytryny
    \item sól
    \item pieprz czarny
\end{Ingred}

Filety z~łososia posiekać w~drobną kostkę. Szalotkę i~kapary pokroić w~drobną
kostkę, dodać koperek, wódkę, sok z~cytryny i~oliwę z~oliwek. Składniki sosy
wymieszać, doprawić solą i~pieprzem. Przygotowanym sosem zalać łososia
i~dokładnie wymieszać. Następnie wstawić do lodówki na około \qty{20}{\minute},
podawać z~grzankami.

\recipe{Nadzienie do schabu w~galarecie}

\begin{Ingred}
    \item \num{1 }duży chrzan
    \item \num{5} gotowanych jajek
    \item \num{.5} kostki masła.
\end{Ingred}

Zetrzeć na dużych oczkach.

\end{document}
