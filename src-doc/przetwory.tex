\documentclass[../main.tex]{subfiles}

\begin{document}

\recipe{Przyprawa do konserwowania ogórków}

\begin{Ingred}
    \item liść laurowy
    \item ziele angielskie
    \item ziele kopru ogrodowego
    \item kminek
    \item korzeń chrzanu
    \item pieprz czarny
    \item cebula
    \item estragon
    \item gorczyca
    \item majeranek
\end{Ingred}

Wsypać łyżkę mieszanki przyprawy do słoja o~pojemności \qty{1}{\litre}. Świeże
ogórki poukładać ściśle w~słoju. Zalać \numrange{8}{10} sztuk słoi zalewą
o~składzie \num{1.5} szklanki octu \qty{10}{\percent}, \qty{4}{\litre} wody,
\num{4} łyżki soli, \num{20} łyżek cukru.

\recipe{Sałatka z~ogórków}

\begin{Ingred}
    \item \qty{4}{\kilo\gram} ogórków
    \item \qty{20}{\deka\gram} pietruszki z~natką
    \item \qty{20}{\deka\gram} cebuli
    \item \num{1} szklanka octu
    \item \num{1} szklanka oliwy
    \item \num{0.5} szklanki cukru
    \item \qty{6}{\deka\gram} (\num{4} płaskie łyżki) cukru
    \item \num{12} sztuk goździków
    \item \num{12} ziarenek pieprzu
    \item \num{12} sztuk ziela angielskiego
\end{Ingred}

Ogórki ze skórką pokroić w~plasterki, marchew i~pietruszkę zetrzeć na tarce
jarzynowej. Natkę i~cebulę pokroić. Zalać oliwą, wymieszać i~odstawić na
\qty{1}{\hour}. Ocet z~przyprawami zagotować i~wystudzić. Nałożyć do słoików
warzywa, zalać zalewą i~pasteryzować \qty{15}{\minute}.

\recipe{Surówka z~kapusty}

\begin{Ingred}
    \item około \qty{2}{\kilo\gram} główka kapusty
    \item \qty{2}{\kilo\gram} kolorowej papryki (nie zielona)
    \item \qty{1}{\kilo\gram} cebuli
    \item \qty{1}{\kilo\gram} marchwi
\end{Ingred}

\begin{Ingred}[zalewa]
    \item \qty{0.5}{\litre} oleju lub oliwy
    \item \qty{.25}{\litre} octu
    \item \num{4} stołowe łyżki soli
    \item \num{13} stołowych łyżek cukru
\end{Ingred}

Kapustę poszatkować, paprykę pokroić w~wąskie paseczki, cebulę pokroić
w~plastry, a~marchew zetrzeć na tarce o~grubych oczkach. Razem wszystko
wymieszać i~zalać zmieszaną wcześniej zalewą. Odstawić to wszystko na
\qty{3}{\hour}, mieszając co jakiś czas. Następnie słoiki razem z~zalewą
pasteryzować około \qtyrange{15}{20}{\minute}.

\recipe{Ostry Ketchup}

\begin{Ingred}
    \item \qty{5}{\kilo\gram} pomidorów
    \item \qty{.5}{\kilo\gram} kwaśnych jabłek
    \item \qty{.5}{\kilo\gram} marchwi
    \item \qty{.5}{\kilo\gram} pietruszki
    \item \qty{.5}{\kilo\gram} selera
    \item \qty{.5}{\kilo\gram} cebuli
    \item \qty{.5}{\kilo\gram} cukru
    \item \num{2} łyżki kminku
    \item \num{5} łyżek soli
    \item \qty{20}{\gram} mielonego pieprzu
    \item \num{5} sztuk goździków
    \item \num{1} łyżeczka cynamonu
    \item \num{4} łyżki octu
\end{Ingred}

Wszystkie warzywa i~owoce zmielić w~maszynce do mięsa, dodać pozostałe
składniki i~wymieszać, zostawić pod przykryciem na dzień. Gotować
\qtyrange{2}{3}{\hour} na małym ogniu, cały czas mieszając. Gorące przetrzeć
przez sito lub chłodne przez sokowirówkę i~ponownie zagotować. Gorący ketchup
wlać do słoików.

\recipe{Ketchup}

\begin{Ingred}
    \item \qty{3}{\kilo\gram} pomidorów
    \item \qty{25}{\deka\gram} cebuli
    \item \num{2} płaskie łyżki soli
    \item \qty{20}{\deka\gram} cukru
    \item \num{.5} szklanki octu \qty{10}{\percent}
    \item \num{1} płaska łyżeczka papryki
    \item \num{1} płaska łyżeczka pieprzu
    \item \num{1} płaska łyżeczka gałki
    \item trochę imbiru
\end{Ingred}

\recipe{Buraczki na zimę}

\begin{Ingred}
    \item \qty{3}{\kilo\gram} buraczków
    \item \num{4} duże cebule
    \item \qty{1}{\kilo\gram} papryki
    \item \num{1} szklanka octu
    \item \num{1} szklanka wody
    \item \num{.5} szklanki cukru
    \item \num{1} łyżka soli
\end{Ingred}

Buraki gotować, aż będą miękkie, zetrzeć na tarce jarzynowej. Zalewę ugotować,
paprykę i~cebulę pokroić w~plastry. Wszystkie składniki wrzucić w~zalewę octową
i~zagotować.

\recipe{Łagodne ogórki konserwowe}

\begin{Ingred}
    \item \num{1} szklanka octu \qty{10}{\percent}
    \item \num{5} szklanek wody
    \item \num{8} łyżek cukru
    \item \num{3} łyżki soli
    \item ogórki
    \item marchew
    \item liść laurowy
    \item pieprz
    \item gorczyca
    \item ziele angielskie
    \item chrzan
    \item koper
\end{Ingred}

Ocet, wodę, cukier oraz sól gotować \qty{5}{\minute}. Pozostałe składniki
włożyć do słoików, zalać zalewą. Pasteryzować przez \qty{2}{\minute} od
zagotowania. Końce ogórków mają być zielone.

\recipe{Papryka konserwowa od Pani Eli Klocek}

\begin{Ingred}
    \item \qty{1.5}{\kilo\gram} papryki
    \item gorczyca
    \item pieprz
    \item liść laurowy
    \item \qty{3}{\litre} wody
    \item \num{2} szklanki octu \qty{10}{\percent}
    \item \num{2} czubate łyżki cukru
    \item \num{2} czubate łyżki soli.
\end{Ingred}

Paprykę pokroić, wrzucić na gorącą wodę, wyjąć i~ułożyć w~słoikach, dodać
przyprawy. Zalewać wrzącą zalewą, na wierzch wlać łyżkę oleju, zamknąć, nie
pasteryzować.

\recipe{Dynia marynowana}

\begin{Ingred}
    \item \qty{1}{\kilo\gram} obranej dyni
    \item \num{3} szklanki octu \qty{3}{\percent}
    \item \num{1.5} szklanki cukru
    \item płaska łyżeczka goździków
    \item kawałek kory cynamonu
\end{Ingred}

Dynię pokroić w~kostkę, na \qty{7}{\minute} włożyć do wrzącej wody i~osączyć na
sicie. W garnku zagotować ocet z~cukrem oraz z~przyprawami, wrzucić kawałki
dyni i~mieszając gotować na wolnym ogniu, aż dynia stanie się szklista.
Napełnić gorącą marynatą słoiczki.

\recipe{Dynia z~rumem}

\begin{Ingred}
    \item średniej wielkości dynia
    \item \qty{1}{\litre} wody
    \item \qty{1}{\litre} octu
    \item \qty{1}{\kilo\gram} cukru
    \item szklanka rumu \qty{40}{\percent}
    \item cynamon
\end{Ingred}

Dynię obrać i~pokroić w~kostkę. Wodę zagotować z~octem oraz cukrem, następnie
wystudzić i~wymieszać z~rumem. Dynię przełożyć do słoików i~do każdego dodać
odrobinę cynamonu, zalać przygotowaną zalewą, zakręcić i~gotować około
\qty{25}{\minute}. Polecana do sałatek z~kurczaka, tuńczyka lub groszku.

\recipe{Dynia Międzyborów}

\begin{Ingred}
    \item \num{1} dynia
    \item \num{4} szklanki cukru
    \item \num{4} szklanki wody
    \item szklanka octu \qty{10}{\percent}
    \item parę goździków
\end{Ingred}

Zagotować zalewę, dynię obrać i~pokroić na kawałki, zagotować w~wodzie z~octem
(\numrange{4}{5} łyżeczek). Gorącą dynię włożyć do słoików i~zalać gorącą
zalewą. Odwrócić do góry dnem.

\recipe{Sok przecierowy z~dyni}

\begin{Ingred}
    \item \qty{5}{\kilo\gram} jabłek
    \item \qty{3}{\kilo\gram} marchewki
    \item \num{2} szklanki cukru
    \item \num{.5} dużej dyni
    \item \num{2} cytryny
\end{Ingred}

Dynię obrać, pokroić w~kostkę, zalać małą ilością wody, dodać \num{.5} szklanki
cukru, gotować. Marchewki obrać i~zetrzeć na tarce o~dużych oczkach. Zalać małą
ilością wody, dodać \num{.5} szklanki cukru, gotować. Dynię, marchewkę i~jabłka
zmiksować. Wlać sok z~cytryny i~wymieszać. Sok wlać do butelek i~gotować około
\qty{5}{\minute}.

\recipe{Dynia w~occie}

\begin{Ingred}
    \item \num{1}szklanka octu
    \item \num{4} szklanki wody
    \item \num{4} szklanki cukru
    \item dynia
\end{Ingred}

Dynię ugotować w~wodzie z~octem.

\recipe{Sos do makaronu}

\begin{Ingred}
    \item \qty{5}{\kilo\gram} pomidorów
    \item \qty{1.5}{\kilo\gram} cebuli
    \item \num{1} łyżka soli
    \item \num{1} papryka czerwona
    \item \num{.5} łyżeczki chilli
    \item \num{1} puszka ananasów w~plasterkach bez soku
    \item \num{1} puszka kukurydzy
    \item \num{.5} łyżeczki czosnku granulowanego lub mielonego
    \item \num{1} łyżeczka „Vegety”
    \item \num{1} łyżeczka oregano
    \item \numrange{1}{2} łyżeczki curry
    \item \num{2} łyżki słodkiej papryki
    \item \num{1} łyżeczka białego pieprzu
    \item \num{1} szklanka octu \qty{10}{\percent}
    \item \num{3} szklanki cukru
    \item \num{2} łyżki mąki ziemniaczanej
\end{Ingred}

Pomidory sparzyć, zdjąć skórkę i~pokroić w~kostkę. Cebulę pokroić w~kostkę.
Pokrojone pomidory i~cebulę włożyć do garnka, dodać łyżkę soli, wymieszać
i~odstawić na noc. Na drugi dzień odcedzić nadmiar sosu. Do otrzymanego miąższu
dodać chilli, czosnek, Vegetę, oregano, curry, musztardę, słodką paprykę,
pokrojonego ananasa i~kukurydzę. Całość zagęścić \num{2} łyżkami mąki
ziemniaczanej rozprowadzonej w~wodzie. Gorący sos wkładać do słoików, odwrócić.

\recipe{Sałatka pieczarkowa}

\begin{Ingred}
    \item \qty{1.5}{\kilo\gram} papryki
    \item \qty{1.5}{\kilo\gram} pieczarek
    \item \qty{.5}{\kilo\gram} cebuli
\end{Ingred}

\begin{Ingred}[zalewa]
    \item \num{1} szklanka wody
    \item \num{1} szklanka octu
    \item \num{1} szklanka oleju
    \item \num{1} szklanka cukru
    \item \num{1} łyżka soli
    \item ziele angielskie
    \item gorczyca
    \item listek laurowy.
\end{Ingred}

Zagotować zalewę, wrzucić pieczarki, gotować przez \qty{5}{\minute}, następnie
dodać paprykę i~cebulę, gotować kolejne \qty{5}{\minute} od zagotowania. Włożyć
do słoików, zalać gorącą zalewą, zakręcić, przekręcić słoiki do góry dnem do
ostygnięcia.

\recipe{Powidła ze śliwek}

\begin{Ingred}
    \item \qty{1.5}{\kilo\gram} śliwek
    \item \qty{1.5}{\kilo\gram} cukru
    \item \num{.25} laski cynamonu
    \item \num{2} łyżki rumu
    \item \num{3} goździki
    \item \num{.25} laski wanilii
\end{Ingred}

Śliwki umyć, dobrze obsuszyć, usunąć pestki i~pokroić w~ćwiartki. Do
\qty{1}{\kilo\gram} śliwek dodać cukier, rum i~przyprawy, dokładnie wymieszać.
Wszystkie składniki pozostawić pod przykryciem do kolejnego dnia, aby owoce
puściły sok. Śliwki zagotować na silnym ogniu, cały czas mieszając drewnianą
łyżką i~utrzymać w~stanie wrzenia przez \qty{4}{\minute}. Zestawić z~ognia
i~wyjąć przyprawy. Słoiki sparzyć wrzącą wodą i~wytrzeć do sucha czystką
ściereczką. Słoiki napełnić gorącymi powidłami, gdy na powierzchni powideł
pojawi się skorupka, należy przykryć słoiki celofanem uprzednio namoczonym
w~rumie, celofan mocno związać.

\recipe{Galaretka agrestowa}

\begin{Ingred}
    \item \qty{850}{\gram} agrestu
    \item \qty{350}{\gram} cukru
    \item \qty{100}{\milli\litre} wody
\end{Ingred}

Cukier zalać wodą, wymieszać, zagotować. Przez kilka minut pozwolić, aby
gotował się syrop cukrowy. Wsypać oczyszczone z~szypułek owoce agrestu.
Przykryć i~gotować przez około \qty{15}{\minute}, co jakiś czas mieszając.
Konfiturę zblendować na najwyższych obrotach i~przetrzeć przez sito, aby pozbyć
się twardych pestek. Powstałą galaretkę przelać do słoików. Mocno zakręcić
i~odstawić do wystygnięcia do góry dnem.

\recipe{Fasolka szparagowa w~sosie pomidorowym}

\begin{Ingred}
    \item \qty{2}{\kilo\gram} żółtej fasolki szparagowej
    \item \qty{40}{\deka\gram} marchewki
    \item \qty{40}{\deka\gram} cebuli
    \item \qty{500}{\milli\litre} przecieru pomidorowego
    \item \num{1} szklanka oleju
    \item \num{2} liście laurowe
    \item \num{6} łyżek cukru
    \item \num{3} łyżki soli
    \item \num{2} łyżki octu \qty{10}{\percent}
    \item pieprz
\end{Ingred}

Fasolkę myjemy, odcinamy końcówki i~kroimy na \qty{3}{\centi\metre} kawałki.
Zalewamy wodą do poziomu fasolki, dodajemy po łyżce soli i~cukru. Gotujemy
około \qty{10}{\minute} i~odcedzamy. Cebulę obieramy i~kroimy w~cienkie
półksiężyce. Marchewkę obieramy i~ścieramy na tarce o~dużych oczkach.

Na patelni podgrzewamy szklankę oleju, wsypujemy cebulę z~marchewką i~dusimy
około \qty{15}{\minute}. Następnie dodajemy liście laurowe, \num{2} łyżki soli,
\num{5} łyżek cukru, trochę pieprzu do smaku i~przecier pomidorowy, a~na końcu
ocet. Mieszamy i~dodajemy podgotowaną fasolkę szparagową. Dusimy wszystko razem
przez \qty{15}{\minute} i~gorące przekładamy do słoików. Zakręcamy
i~pasteryzujemy około \qty{20}{\minute}.

\recipe{Fasolka szparagowa w~sosie pomidorowym \rom{2}}

\begin{Ingred}
    \item \qty{2}{\kilo\gram} fasolki szparagowej
    \item \qty{40}{\deka\gram} cebuli
    \item \qty{40}{\deka\gram} marchewki
    \item \num{2} papryki
    \item \num{2} słoiczki koncentratu
    \item \num{3} łyżki soli
    \item \num{6} łyżek cukru
    \item \num{6} łyżek octu
    \item \num{1} szklanki oleju
    \item \num{4} ząbki czosnku
    \item pieprz
\end{Ingred}

Na patelni w~oleju dusimy cebule i~marchew, w~końcowej fazie dodajemy pokrojoną
w paseczki paprykę wymieszaną z~solą i~cukrem. Fasolę pokrojoną na
\qtyrange{2}{3}{\centi\metre} kawałki, gotujemy w~oddzielnym garnku (od
zawrzenia \qty{13}{\minute}). Po ugotowaniu odcedzamy, mieszamy z~resztą
warzyw, chwilę dusimy. Pod koniec duszenia wyciskamy czosnek przez praskę.
Gorące warzywa wkładamy w~słoiki i~pasteryzujemy około \qty{15}{\minute}.

\end{document}
