\documentclass[main.tex]{subfiles}
\begin{document}

\recipe{Pierś indycza w~sosie śmietanowo-pieczarkowym}

\begin{Ingred}
    \item \num{1} pierś indycza
    \item \num{2} łyżki soku z~cytryny
    \item \qty{150}{\gram} pieczarek
    \item \qty{300}{\milli\litre} bulionu
    \item \qty{100}{\milli\litre} śmietanki \qty{30}{\percent}
    \item sok z~\num{1} cytryny
    \item olej
    \item sól
    \item pieprz czarny mielony
    \item \num{.5} łyżki pietruszki
\end{Ingred}

Pierś indyczą natrzeć solą, pieprzem, następnie skropić sokiem z~cytryny
i~odstawić na \qtyrange{3}{4}{\hour} w~chłodne miejsce. Na patelni rozgrzać
olej, pierś obsmażyć z~każdej strony, przełożyć do wysmarowanego olejem
naczynia żaroodpornego. Piec pod przykryciem w~temperaturze \qty{200}{\celsius}
około \qtyrange{50}{60}{\minute}. Pod koniec pieczenia odkryć mięso, aby się
zarumieniło.

Oczyszczone, umyte pieczarki pokroić na plasterki i~skropić sokiem z~cytryny,
żeby nie ściemniały. Cebulę obrać, posiekać i~udusić z~pieczarkami na maśle.
Osolić, oprószyć mąką, wlać bulion i~śmietankę. Wymieszać, dodać natkę
i~zagotować. Posypać zieloną pietruszką. Pierś podawać pokrojoną w~plastry
z~sosem pieczarkowym.

\recipe{Szparagi w~cieście naleśnikowym}

\begin{Ingred}[dla czterych osób]
    \item \qty{1.5}{\kilo\gram} białych szparagów
    \item \num{.5} szklanki przesianej mąki
    \item \num{.5} szklanki mleka
    \item \num{3} jajka
    \item \num{2} łyżki startego parmezanu
    \item łyżka masła
    \item \num{8} plastrów szynki
    \item \qty{200}{\gram} jogurtu naturalnego
    \item \num{1} łyżeczka startego chrzanu
    \item \num{2} łyżki śmietany
    \item \num{1} pęczek natki pietruszki
    \item olej
    \item cukier
    \item sól
    \item pieprz czarny mielony
\end{Ingred}

Mąkę wymieszać z~mlekiem, dodać jajka, parmezan i~stopione masło, stale
mieszając. Wyrobić gładkie ciasto i~odstawić na około \qty{30}{\minute}.

Obrane szparagi gotować \qtyrange{15}{20}{\minute} w~osolonej wodzie
z~dodatkiem łyżeczki cukru. Wyjąć z~wody i~odsączyć. Usmażyć na oleju \num{8}
cienkich naleśników. Jogurt wymieszać z~solą, pieprzem, chrzanem, śmietaną
i~posiekaną natką pietruszki. Na każdym naleśniku ułożyć plaster szynki
i~\numrange{4}{5} szparagów, zawinąć, podawać polane sosem z~zieloną
pietruszką.

\recipe{Chleb razowy z~ziarnami}

\begin{Ingred}
    \item \qty{.5}{\kilo\gram} mąki pszennej
    \item \qty{.25}{\kilo\gram} mąki żytniej razowej
    \item \num{.5} szklanki pestek dyni
    \item \num{.5} szklanki ziaren słonecznika
    \item \num{.5} szklanki otrąb
    \item \num{.5} szklanki siemienia lnianego
    \item \num{.5} szklanki płatków owsianych
    \item łyżka soli
    \item łyżka oliwy
    \item zakwas
\end{Ingred}

Wymieszać sypkie składniki w~misce, dodać olej, sól, zakwas i~\num{3.5}
szklanki ciepłej wody. Dobrze wymieszać łyżką. Odłożyć \num{3} łyżki zakwasu do
słoika. Wysmarować długą wąską blachę smalcem. Włożyć ciasto i~zostawić na noc
do wyrośnięcia. Piec \qty{70}{\minute} w~temperaturze \qty{200}{\celsius}.
Gorący chleb wyjąć na deskę do góry nogami.

\recipe{Parówki na boczku}

\begin{Ingred}
    \item \num{10} parówek
    \item \num{10} plastrów chudego boczku
    \item \qty{15}{\deca\gram} żółtego sera
    \item słodka papryka
    \item majeranek
\end{Ingred}

Parówki obrać z~osłonek i~ponacinać wzdłuż, ale nie do końca. Ser pokroić
w~słupki o~szerokości około \qty{1}{\centi\metre}. Każdą parówkę oprószyć
w~środku roztartym majerankiem i~papryką. Włożyć kawałek sera do parówki
i~następnie owinąć plasterkiem boczku i~przekłuć wykałaczką. Parówki zapiekać
w~gorącym piekarniku do chwili, gdy boczek się zarumieni, a~ser się roztopi.

\recipe{Kurczak duszony}

\begin{Ingred}
    \item \qty{600}{\gram} mięsa z~kurczaka
    \item szklanka jogurtu naturalnego
    \item łyżka masła
    \item \num{4} łyżeczki \enquote{Przyprawy do kurczaka po staropolsku}
\end{Ingred}

Mięso z~kurczaka pokroić w~kostkę i~włożyć do jogurtu wymieszanego z~przyprawą.
Odstawić na \qtyrange{1}{2}{\hour} do lodówki. W garnku rozpuścić masło
i~przełożyć do niego kurczaka wraz z~sosem. Dusić pod przykryciem około
\qty{15}{\minute}. Następnie zdjąć przykrywkę i~gotować na małym ogniu do
częściowego odparowania sosu.

\recipe{Pomidory z~serem i~salami}

\begin{Ingred}
    \item \num{10} średnich pomidorów
    \item \qty{15}{\deca\gram} sera rokpola
    \item \qty{15}{\deca\gram} salami w~plasterkach
    \item \numrange{4}{5} małych ogórków marynowanych
    \item \num{1} pęczek szczypiorku
    \item \num{1} łyżka posiekanych orzechów włoskich
    \item \num{1} łyżeczka tartego chrzanu
    \item \num{2} łyżki jogurtu
    \item \num{2} łyżki majonezu
    \item sól
    \item papryka ostra mielona
\end{Ingred}

Pomidory umyć, odkroić wierzchołki. Dolne części wydrążyć, zostawiając
\qty{.5}{\centi\metre} warstwę miąższu przy samej skórce. Ser rozgnieść
widelcem, dodać miąższ z~pomidorów, pokrojoną wędlinę, ogórki i~posiekany
szczypiorek. Całość wymieszać z~orzechami, chrzanem jogurtem oraz majonezem.
Dodać nieco mielonej papryki i~soli. Farszem wypełnić pomidory, przykrywając je
odkrojonymi czapeczkami.

\recipe{Zapiekanka rybna}

\begin{Ingred}
    \item \num{5} ziemniaków
    \item \num{4} cebule
    \item \num{.5} szklanki białego wytrawnego wina
    \item łosoś
    \item sól
    \item pieprz
    \item śmietana
    \item czosnek
    \item \qty{600}{\gram} owoców morza
    \item żółty ser
\end{Ingred}

Ziemniaki ugotować i~pokroić w~talarki, cebule drobno pokroić. Wino zagotować,
wsypując do niego cebulę. Doprawionego łososia cienko pokroić. Śmietanę wlać do
wina. Owoce morza posolić, popieprzyć i~cienko pokroić, czosnek również drobno
pokroić. Ser zetrzeć na dużych oczkach. Piec wszystko w~\qty{180}{\celsius}.

\recipe{Omlet z~bazylią}

\begin{Ingred}
    \item \num{2} jajka
    \item łyżeczka bazylii
    \item \num{3} łyżki mleka
    \item \num{1} łyżka masła
    \item pomidor
    \item sól
    \item pieprz
\end{Ingred}

Bazylię namoczyć w~mleku. Pomidora sparzyć, obrać ze skórki, pokroić na
plasterki. Jaja roztrzepać widelcem jednocześnie dodając namoczoną bazylię,
doprawić solą i~pieprzem. Na patelni rozgrzać masło, wlać masę jajeczną,
smażyć. Pod koniec smażenia nałożyć pomidory na połowę omleta.

\recipe{Polędwiczki wieprzowe w~sosie prowansalskim}

\begin{Ingred}
    \item \qty{600}{\deca\gram} polędwiczek wieprzowych
    \item \num{1} łyżka masła
    \item \num{1} ząbek czosnku
    \item \num{.5} szklanki białego wytrawnego wina
    \item \num{2} łyżki śmietany
    \item \num{1} żółtko
    \item \num{3} korniszony
    \item \num{1} łyżeczka ziół prowansalskich
    \item \num{2} łyżki musztardy krymskiej
    \item oliwa z~oliwek
    \item sól
    \item mielony pieprz czarny
\end{Ingred}

Polędwiczki przysmażyć na oliwie z~masłem z~każdej strony, dodać rozgnieciony
czosnek, chwilę smażyć. Posypać solą i~pieprzem, odstawić w~ciepłe miejsce do
pozostałego na patelni tłuszczu dodać białe wino i~\num{3} łyżki gorącej wody,
wymieszać i~zagotować. Śmietanę, musztardę i~żółtko ubić trzepaczką, dodać do
sosu i~dalej ubijać, aż do utrzymania jednolitej konsystencji. Zagotować, dodać
zioła prowansalskie i~pokrojone korniszony.

\recipe{Ciasto na pizzę oraz sos pomidorowy}

\begin{Ingred}[ciasto]
    \item \qty{500}{\gram} mąki pełnoziarnistej
    \item \qty{20}{\gram} drożdży
    \item \num{.5} łyżeczki soli
    \item \qty{.25}{\litre} letniej wody
\end{Ingred}

\begin{Ingred}[sos pomidorowy]
    \item puszka pomidorów (około \qty{850}{\milli\litre})
    \item \num{2} ząbki czosnku
    \item \num{2} łyżki oleju
    \item \num{.5} łyżeczki tymianku
    \item sól
    \item pieprz
\end{Ingred}

\paru{Ciasto} W mące zrobić zagłębienie, umieścić w~nim rozdrobnione drożdże,
wymieszać z~\qty{.25}{\litre} letniej wody i~odrobiną mąki. Odczekać
\qty{15}{\minute}. Ciasto posolić i~zagniatać, aż do utworzenia pęcherzyków
powietrza, odstawić do wyrośnięcia. Pizza z~białej mąki jest twardsza, a~z mąki
pełnoziarnistej chrupiąca. Na posypanej stolnicy rozwałkować ciasto na placek
i~przełożyć na wysmarowaną tłuszczem blachę. Posmarować sosem pomidorowym
i~dodać resztę składników. Wstawić do piekarnika i~piec około
\qtyrange{20}{25}{\minute} w~temperaturze \qty{225}{\celsius}.
\paru{Sos pomidorowy} Pomidory pokroić w~ćwiartki, przełożyć na sito. Czosnek
pokroić w~kostkę i~poddusić na oleju. Dodać pomidory, tymianek i~wymieszać.
Gotować kilka minut, dolewając sok z~pomidorów, aby sos był gęsty. Gotowy sos
doprawić do smaku solą i~pieprzem.

\recipe{Wałeczki z~kurczaka z~sosem tatarskim}

\begin{Ingred}
    \item \num{2} piersi kurczaka
    \item \num{.5} strąka papryki pokrojonej w~kostkę
    \item pęczek posiekanej natki pietruszki
    \item białko z~jajka
    \item \num{3} łyżki bułki tartej
    \item \qty{50}{\milli\litre} oleju rzepakowego
    \item sól
    \item pieprz
\end{Ingred}

\begin{Ingred}[sos tatarski]
    \item \qty{150}{\gram} majonezu
    \item pęczek posiekanego szczypiorku
    \item pęczek posiekanego koperku
    \item \qty{100}{\gram} pokrojonych w~kostkę ogórków kwaszonych
\end{Ingred}

\paru{Sos tatarski} Majonez wymieszaj z~pozostałymi składnikami, dopraw do smaku
solą i~pieprzem, odstaw do ostygnięcia do lodówki.
\paru{Wałeczki} Pierś z~kurczaka zmielić w~maszynce. Dodać paprykę, natkę
pietruszki, białko, łyżkę bułki tartej oraz sól i~pieprz. Wyrobić na jednolitą
masę i~uformować wałeczki wielkości serdelków. Mięso obtoczyć w~bułce tartej
i~usmażyć ze wszystkich stron na rozgrzanym oleju. Wałeczki podawać z~sosem
tatarskim i~pieczonymi ziemniakami.

\recipe{Łopatka wieprzowa duszona w~sosie barbecue}

\begin{Ingred}
    \item opakowanie łopatki wieprzowej \enquote{Kraina mięsa}
    \item \qty{100}{\milli\litre} bulionu z~kostki drobiowej
    \item pokrojona w~plastry cebula
    \item \num{2} posiekane ząbki czosnku
    \item \num{1} puszka pomidorów w~zalewie
    \item \qty{200}{\gram} konfitury śliwkowej
    \item marchewka
    \item pietruszka
    \item kilka liści laurowych
    \item łyżka cukru
    \item \numrange{1}{2} łyżki octu spirytusowego
    \item sól
    \item pieprz
\end{Ingred}

Marchewkę i~pietruszkę pokroić w~około \qty{3}{\centi\metre} kawałki. Łopatkę
doprawić solą, czosnkiem, pieprzem, obłożyć zielem angielskim i~odstawić na
\qty{1}{\hour} w~chłodne miejsce. Mięso włożyć do naczynia do zapiekania. Na
łopatce ułożyć cebulę, marchew, pietruszkę i~selera. Konfiturę wymieszać
z~pomidorami oraz bulionem i~polać tym mięso. Naczynie przykryć i~wstawić do
nagrzanego do \qty{190}{\celsius} piekarnika na \qty{2}{\hour}. Po tym czasie
wyjąć mięso z~naczynia i~pokroić w~plastry. Z sosu wyjąć do dekoracji kawałki
marchewki i~pietruszki, a~resztę zmiksować na gładki sos z~dodatkiem cukru
i~octu. Mięso podawać z~warzywami i~sosem.

\recipe{Pierogi dyniowe}
\begin{Ingred}[ciasto]
    \item \num{3} szklanki mąki
    \item \num{1} jajko
    \item \numrange{.5}{.66} szklanki wody
    \item szczypta soli
\end{Ingred}

\begin{Ingred}[farsz]
    \item \qty{200}{\gram} upieczonej dyni Hokkaido
    \item około \qty{250}{\gram} twarogu
    \item \num{2} łyżki miodu lub syropu z~agawy
    \item garść rodzynek
\end{Ingred}

Pokroić dynie na \num{4} części, wyjąć pestki. Wstawić do piekarnika
ustawionego na \qty{180}{\celsius} i~piec około \qty{30}{\minute}. Następnie
zmiksować. Po upieczeniu skórka ładnie odchodzi.

Upieczoną dynię wymieszać z~twarożkiem, dosłodzić i~zmiksować na gładko, dodać
rodzynki. Składniki na ciasto połączyć, następnie wyrabiać do uzyskania
gładkiej kuli. Odłożyć na \qty{10}{\minute}, podzielić na mniejsze części
i~wałkować na blacie obsypanym mąką. Kieliszkiem do szampana wycinać kółka. Na
połowę kółek nałożyć farsz, drugą częścią kółek dopasować przykrywając farsz.
Gotowe pierogi wrzucić na gotującą, osoloną wodę i~gotować około
\qtyrange{4}{5}{\minute} od ponownego zagotowania. Podawać ze śmietaną lub
polane syropem z~dyni lub klonowym.

\recipe{Pierogi z~dynią}

\begin{Ingred}[ciasto]
    \item \qty{300}{\gram} mąki pszennej
    \item \qty{250}{\milli\litre} ciepłej wody
    \item \num{.25} łyżeczki soli
\end{Ingred}

\begin{Ingred}[farsz]
    \item \qty{400}{\gram} miąższu dyni
    \item \num{1} cebula
    \item \num{2} ząbki czosnku
    \item \num{1} szklanka mrożonego zielonego groszku
    \item sól
    \item pieprz
    \item imbir
    \item kurkuma
    \item suszony imbir
    \item cynamon
    \item płatki chilli
\end{Ingred}

Dynie pozbawić pestek i~skóry, pokroić w~kostkę. W dużej misce kawałki dyni
posolić, popieprzyć i~skropić oliwą. Dynię wrzucić na blachę i~wstawić na
\qty{20}{\minute} do piekarnika nagrzanego do \qty{200}{\celsius}. Mąkę
przesiać do dużej miski, dodać sól, następnie wolno dolewając wodę, wymieszać
całość. Ciasto wyrobić. Miskę z~ciastem przykryć ściereczką i~pozostawić na
\qty{30}{\minute}. Cebulę pokroić w~cienkie piórka, usmażyć na złoty kolor.
Kiedy cebula jest już prawie gotowa dodać posiekany czosnek. Połowę usmażonej
cebuli odstawić, można nią pokrasić później pierogi. Upieczoną dynię przełożyć
do miski, dodać groszek, połowę cebuli i~starty imbir. Farsz doprawić do smaku
solą, pieprzem i~kurkumą, dodatkowo można doprawić suszonym imbirem, cynamonem
i płatkami chilli. Cały farsz dokładnie wymieszać. Odlać nadmiar soku z~farszu.
Ciasto cienko rozwałkować, wyciąć kółka za pomocą szklanki, nałożyć farsz
i~skleić pierogi. Wrzucić do gotującej, lekko osolonej wody. Wyjmować po
kilkudziesięciu sekundach od wypłynięcia. Pierogi można posmarować masłem
i~posypać pozostałą przysmażoną cebulką oraz szczypiorkiem.

\recipe{Pieczona biała kiełbasa}

\begin{Ingred}
    \item \qty{1}{\kilo\gram} surowej białej kiełbasy
    \item \num{2} cebule
    \item \qty{150}{\gram} parzonego lub wędzonego boczku
\end{Ingred}

\begin{Ingred}[zalewa]
    \item \num{.25} szklanki piwa lub białego wina lub \num{2} łyżki octu
    \item \num{3} łyżki musztardy francuskiej ziarnistej
    \item \num{2} łyżki sosu sojowego
    \item \num{1} łyżeczka miodu
    \item \num{1} łyżeczka suszonego majeranku
    \item \num{.25} łyżeczki zmielonego pieprzu
\end{Ingred}

Piekarnik nagrzać do \qty{180}{\celsius}. Do naczynia żaroodpornego włożyć
kiełbasę. Cebulę obrać, przekroić na połówki i~pokroić na plasterki, dodać do
kiełbasy. Składniki zalewy wymieszać w~miseczce. Polać po kiełbasie, następnie
posypać pokrojonym w~kosteczkę boczkiem. Naczynie przykryć folią aluminiową lub
pokrywą i~piec przez \qty{45}{\minute}, następnie zdjąć przykrycie i~piec
jeszcze przez kolejne \qty{45}{\minute}. W razie potrzeby na koniec można
ustawić funkcję grilla i~zrumienić kiełbasę, piekąc ją przez kilka dodatkowych
minut.

\end{document}
